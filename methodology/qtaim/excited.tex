% -*- coding: utf-8 -*-
\subsection{Excited States}

The study of electronic excited states plays an essential role to
understand the spectroscopic and photochemical properties of molecular
systems. Following the approach adopted in previous sections ---where molecular
properties such as dipole moments and polarisabilities were decomposed into
atomic contributions via QTAIM--- this subsection will similarly progress from
global characterisation of electronic excitations toward an atomic-level
interpretation facilitated by QTAIM partitioning.

We will restrict our focus here to vertical excitations, defined as electronic
transitions occurring without nuclear displacement. This approximation aligns
with the Franck–Condon principle~\cite{Franck1926, Condon1926}, assuming fixed
nuclear geometry for both ground and excited states, and is widely employed in
theoretical treatments of electronic excitations.

\newpage
To characterise the nature of these excitations, it is insightful to examine
the redistribution of the electron density between the excited
$\rho_{\mathrm{EX}}(\mathbf{r})$ and ground states
$\rho_{\mathrm{GS}}(\mathbf{r})$. This redistribution can be described
quantitatively by the electron density difference:

\begin{align}
  \Delta\rho (\mathbf{r}) = \rho_{\mathrm{EX}}(\mathbf{r}) - \rho_{\mathrm{GS}}(\mathbf{r}).
\end{align}

The density difference $\Delta\rho(\mathbf{r})$ provides a spatial map of the
electron redistribution upon excitation. Regions where $\Delta\rho(\mathbf{r})
> 0$ correspond to electron accumulation, while regions where
$\Delta\rho(\mathbf{r}) < 0$ indicate electron depletion. To isolate these
contributions, it is useful to define the positive and negative components of
the density difference, denoted $\rho_+$ and $\rho_-$, respectively. These
functions distinguish between areas where electronic charge has been gained
(typically interpreted as particle-like) and lost
(hole-like)~\cite{Gatti2022,Herbert2023}.

\vspace{1em}%
\dfn{$\rho_+$ \& $\rho_-$}{
  % The positive and negative parts of the density difference are defined as
  %
  \begin{align}
    \rho_+ (\mathbf{r}) =
    \begin{cases} 
      \Delta\rho(\mathbf{r}) & \text{if } \Delta\rho(\mathbf{r}) > 0\\
      0     & \text{otherwise}
    \end{cases}
    \label{rho_plus}
  \end{align}
  %
  \begin{align}
    \rho_- (\mathbf{r}) =
    \begin{cases} 
      \Delta\rho(\mathbf{r}) & \text{if } \Delta\rho(\mathbf{r}) < 0\\
      0     & \text{otherwise}
    \end{cases}
    \label{rho_minus}
  \end{align}

}
\vspace{1em}%

A useful measure of charge separation can be obtained by computing the
centroids of the $\rho_+$ and $\rho_-$ distributions. These centroids,
denoted $\mathbf{R}_+$ and $\mathbf{R}_-$, correspond to the average positions
of the regions where electron density is accumulated and depleted,
respectively.

\begin{align}
  \mathbf{R}_\pm = \frac
    {\int \mathbf{r}\rho_{\pm}(\mathbf{r})\dd\tau}
      {\int\rho_{\pm}(\mathbf{r})\dd\tau} =
    (x_{\pm}, y_{\pm}, z_{\pm}).
  \label{centroid}
\end{align}

\newpage
The spatial extent of charge transfer upon excitation can be estimated by the
Euclidean distance between the centroids $\mathbf{R}_+$ and $\mathbf{R}_-$:

\begin{align}
  D_\mathrm{CT} = \norm{\mathbf{R}_+ - \mathbf{R}_-}.
  \label{ct_exci}
\end{align}

This distance characterises how far the electron density is displaced during
the excitation. To quantify the magnitude of the charge transferred, we define
$q_{\mathrm{CT}}$ as the integrated value of either $\rho_+$ or $\rho_-$:

\begin{align}
  q_{\mathrm{CT}} = \int\rho_+(\mathbf{r})\dd\tau = -\int\rho_-(\mathbf{r})\dd\tau.
\end{align}

Together, $D_\mathrm{CT}$ and $q_\mathrm{CT}$ provide a compact and physically
meaningful expression for the change in dipole moment associated with the
charge transfer process:

\begin{align}
  \norm{\Delta\boldsymbol{\mu}_\mathrm{CT}} = D_\mathrm{CT} \cdot q_{\mathrm{CT}}.
  \label{mu_ct}
\end{align}

% \noindent where $\hat{u}$ is the unit vector pointing from the centroid of the
% hole to the centroid of the electron density. This formulation offers an
% intuitive interpretation of the excitation in terms of the displacement of
% charge in space.

A more detailed link between orbital transitions and real-space charge
redistribution can be established by expressing $\rho_+$ and $\rho_-$ in terms
of natural transition orbitals (NTOs). Within this framework, $\rho_+$ and
$\rho_-$ correspond to the particle (electron) and hole densities, respectively,
each constructed from weighted contributions of individual NTOs. This
representation connects the orbital nature of the excitation with physically
observable quantities such as the dipole moment change and the charge-transfer
distance $D_\mathrm{CT}$.

% Reference to the notation
\newpage
\follow{Derivation of $\boldsymbol{\mu}_{\mathrm{CT}}$}{

  knowing $\rho^e = \sum\lambda(\phi^{IO})^2$ and $\rho^h =
  \sum\lambda(\phi^{OI})^2$, \\
  where $\phi$ denotes the natural transition molecular orbitals,\\ [-1em]
  % \textit{IO} = initial occupied, \textit{OI} = final occupied, \\
  {\tiny
  \begin{flushright}
    \textit{all integrals run over $\dd\tau$, the differentials are omitted for readability.\\
    Jacobi's delta is used to indicate a particular differential
    within a integral, $\int_{\partial x} f(x,y) = \int f(x,y)\dd x$.\\
    The sign function returns $-1$, $0$, or $1$ depending on whether $x$ is negative, zero, or positive, respectively.}
  \end{flushright}}
  % \vspace{-1em}%
  \begin{align}
    \norm{\Delta\boldsymbol{\mu}_\mathrm{CT}} &= D_{\mathrm{CT}} \cdot q_{\mathrm{CT}} =
      D_{\mathrm{CT}}\int\rho_+ = - D_{\mathrm{CT}}\int\rho_-\\
      &= \norm{\mathbf{R}_+ - \mathbf{R}_-}\int\rho_+ =
        \norm{\frac{\int \mathbf{r}\rho_+}{\int\rho_+} - \frac{\int \mathbf{r}\rho_-}{\int\rho_-}} \int\rho_+ =
        \norm{\frac{\int \mathbf{r}\rho_+}{\int\rho_+} + \frac{\int \mathbf{r}\rho_-}{\int\rho_+}} \int\rho_+ \nonumber\\
      &= \norm{\int \mathbf{r}\rho_+ + \int \mathbf{r}\rho_-}\frac{\int\rho_+}{\norm{\int\rho_+}} =
        \norm{\int \mathbf{r}\rho_+ + \int \mathbf{r}\rho_-}\mathrm{sign}\left(\int\rho_+\right)
  \end{align}
  
  with the notation $\rho_+ = \rho^e$ and $\rho_- = \rho^h$, we can write the most explicit form
  of the dipole moment change as

  \tiny{
  \begin{align}
    \norm{\Delta\boldsymbol{\mu}_{\mathrm{CT}}} &= \norm{\int \mathbf{r}\sum\lambda_i^2\abs{\psi_i^e}^2 +
      \int \mathbf{r}\sum\lambda_i^2\abs{\psi_i^h}^2}
      \mathrm{sign}\left(\int\sum\lambda_i^2\abs{\psi_i^e}^2\right) \\
                   &= \norm{\int_{\partial \mathbf{r}^e}\mathbf{r}^e\sum_i\lambda_i\left[\sum_p U_{pi}^e\phi_p(\mathbf{r}^e)\right]^2
    -\int_{\partial \mathbf{r}^h}\mathbf{r}^h\sum_i\lambda_i\left[\sum_p V_{pi}^h\phi_p(\mathbf{r}^h)\right]^2} \cdots\nonumber\\
    &\phantom{= }\cdots\times\mathrm{sign}
    \left(\int_{\partial \mathbf{r}^e}\sum_i\lambda_i^2\left[\sum_p U_{pi}\phi_p(\mathbf{r}^e)\right]^2\right)
  \end{align}
  }

}

Various analyses of excited states rely on molecular orbitals and associated
descriptors, such as the charge-transfer distance ($\Delta\mathbf{r}$) and the
spatial overlap of a given excitation ($\Lambda$)~\cite{Peach2008,Guido2013}.
Equations~\ref{ctdistance} and~\ref{overlapes}, where $K_{ia} = X_{ia} +
Y_{ia}$, account for both excitation ($X_{ia}$) and de-excitation ($Y_{ia}$)
coefficients.

\begin{align}
  \Delta r &= \frac{\sum_{ia}K_{ia}^2\abs{\langle\phi_a|r|\phi_a\rangle -
    \langle\phi_i|r|\phi_i\rangle}}{\sum_{ia}K_{ia}^2} &
  \Lambda  &= \frac{\sum_{ia}K_{ia}^2\langle\abs{\phi_i}\abs{\phi_a}\rangle}{\sum_{ia}K_{ia}^2} \label{ctdistance}\\
  \Delta r &= \frac{\sum_{ia}K_{ia}^2\abs{\int r\phi_a^2 -
    \int r\phi_i^2}}{\sum_{ia}K_{ia}^2} &
  \Lambda  &= \frac{\sum_{ia}K_{ia}^2\int\abs{\phi_i}\abs{\phi_a}}{\sum_{ia}K_{ia}^2} \label{overlapes}
\end{align}

\newpage
Since the integrals involved are defined over the entire spatial domain,
real-space partitioning schemes ---such as that used in \gls{QTAIM}--- 
can apply the same expressions by simply restricting the integration to individual atomic basins.
Consequently, in a code that already supports $\Delta\mathbf{r}$ and $\Lambda$,
such as \adf,
extending the analysis to atomic contributions would require only modest additional effort.

By integrating the functions $\rho_+$ and $\rho_-$ over each basin, we obtain
atomic-level descriptors for electron depletion and accumulation, respectively:

\begin{align}
  q_\pm^\mathrm{CT}(\Omega) &= \int_{\Omega} \rho_\pm(\mathbf{r})\dd\mathbf{r}.
\end{align}

\noindent where $\Omega$ denotes the spatial domain associated with an atomic
basin. These values indicate the amount of electron density gained or lost by
each atom during the transition, enabling a more detailed interpretation of the
excitation in chemically meaningful terms.

Conceptually, this atomic decomposition follows the same logic as standard
population analyses, where global quantities ---such as the total charge or the
dipole moment--- are partitioned into atomic contributions. The total integrals
remain unchanged; only their domains are split according to the real-space
partitioning.



