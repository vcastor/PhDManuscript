% -*- coding: utf-8 -*-
\section{Quantum Mechanics Foundations}

\vspace{0.5cm}%
\begin{flushright}
  {\small
  ``{\em I think I can safely say that nobody
         understands quantum mechanics}''.

  -Richard Feynman
  }
\end{flushright}

\vspace{0.4cm}%
\noindent In this section, we take as our starting point the well-known
Schrödinger Equation~\cite{Schrdinger1926}, postulated by Erwin Schrödinger in
1925. This equation governs the quantum dynamics of a single particle of mass
$m$ evolving under the influence of a potential $V$.

\vspace{0.3cm}%
\dfn{Schrödinger Equation}{
  \begin{equation}
    \gls{iunit}\gls{hbar}\pdv{}{t}\ket{\gls{Psi}(\mathbf{r},t)} =
      \left [ \frac{-\hbar^2}{2m}\nabla^2
      + \widehat{V}(\mathbf{r},t) \right ] \gls{Psiket} =
    \gls{hamiltoniano}\ket{\Psi(\mathbf{r},t)}
    \label{sch_eq}
  \end{equation}
}
\vspace{0.3cm}%

In cases where the Hamiltonian does not depend explicitly on time, 
we may apply the method of separation of
variables. This allows us to factor the wavefunction as
$\Psi(\mathbf{r},t) = e^{-\sfrac{iEt}\hbar} \psi(\mathbf{r})$ and
derive the time-independent Schrödinger equation:

\vspace{0.3cm}%
\keq{Time-Independent Schrödinger Equation}{
  \begin{equation}
    \gls{hamiltoniano} \ket{\psi(\mathbf{r})} = E \ket{\psi(\mathbf{r})},
    \label{time_indep_sch}
  \end{equation}

  \noindent where $E$ denotes the energy eigenvalue associated with the
  stationary state $\ket{\psi}$.
}

\vspace{0.3cm}%
While the Schrödinger equation offers a fundamental description of
non-relativistic quantum systems, a complete treatment must incorporate
relativistic effects, using the Dirac equation~\cite{Dirac1928}.
Rather than delving into the complexities of relativistic quantum mechanics, we
focus on the challenges posed by molecules with many electrons. These systems
are particularly difficult to solve, as no analytic solutions are available
using current state-of-the-art mathematical techniques.

% these are captured by the Dirac equation
% (Equation~\ref{dirac_eq}):

% \nt{
%   \begin{equation}
%     i\hbar\gamma^\mu\partial_\mu\psi - mc\psi = 0
%     \label{dirac_eq}
%   \end{equation}

%   \noindent where the $\gamma^\mu$ are the Dirac gamma matrices, which ensure
%   Lorentz covariance. The equation encodes the structure of spacetime via the
%   four-gradient $\partial_\mu$ and relates energy and momentum through the
%   invariant relation $E^2 = p^2c^2 + m^2c^4$.
% }

\newpage
The complexity and intractability of solving many-electron molecular systems
necessitate the use of approximations. One of the earliest and most fundamental
of these is the Born-Oppenheimer Approximation (\gls{BOA}). Although the BOA is
inherently contradictory to the principles of quantum mechanics, it remains a
powerful tool due to the significant mass difference between nuclei and
electrons.

\vspace{0.5cm}%
% Born-Oppenheimer Approximation
\follow{Born-Oppenheimer Approximation}{

  A full quantum mechanical description of a time-independent system is given by
  the time-independent Schrödinger equation:

  \begin{equation}
    \Ha \ket{\psi} = E \ket{\psi},
  \end{equation}

  where the total energy $E$ can be separated into nuclear and electronic
  contributions:

  \begin{equation}
    E = E_n + E_e.
  \end{equation}

  The total Hamiltonian can then be expressed as:

  \begin{equation}
    \Ha = \Ha_n + \Ha_e = \Ha_{nn} + \Ha_{ne} + \Ha_{ee}.
  \end{equation}

  Here, the $n$-$n$ interactions are treated classically. The electronic
  Schrödinger equation thus takes the form:

  \begin{equation}
    \Ha_e \ket{\psi_e} = E_e \ket{\psi_e}.
  \end{equation}

  As a result, the total energy can be written as:

  \begin{equation}
    E = E_e + E_{nn},
  \end{equation}

  where $E_{nn}$ corresponds to the classical Coulomb repulsion between nuclei.
  At this point, we ``\textit{just}'' need to solve the electronic Schrödinger
  equation. The electronic Hamiltonian includes the kinetic energy of the
  electrons, the electron-nuclear attractions, and the electron-electron
  repulsions:

  \begin{equation}
    \Ha_e = -\sum_i \sfrac{1}{2}\nabla_i^2 - \sum_{i,\alpha}\frac{Z_\alpha}{|\mathbf{R}_\alpha - \mathbf{r}_i|}
      + \sum_{i>j} \frac1{|\mathbf{r}_i - \mathbf{r}_j|},
  \end{equation}

  where the $\alpha$ index runs over nuclei, and the $i,j$ indices run over
  electrons. The Hamiltonian is expressed in atomic units, i.e.\ $\hbar = m_e = e = 1$.

}


This approximation is generally robust, as discussed by \v{C}urík and
Sutcliffe in \cite{urk2018} and \cite{Sutcliffe}. It allows us to
simplify the problem by treating nuclei as stationary, quasi-classical
particles relative to the much lighter and faster-moving electrons, which are
the primary particles of interest in our study.

% ------------------------------
% Include Hartree-Fock Formalism
\subimport{./}{hartree_fock.tex}

% ------------------------------
% Include Basis Sets
\subimport{./}{basis_set.tex}

