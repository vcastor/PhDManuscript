% Born-Oppenheimer Approximation
\follow{Born-Oppenheimer Approximation}{

  A full quantum mechanical description of a time-independent system is given by
  the time-independent Schrödinger equation:

  \begin{equation}
    \Ha \ket{\psi} = E \ket{\psi},
  \end{equation}

  where the total energy $E$ can be separated into nuclear and electronic
  contributions:

  \begin{equation}
    E = E_n + E_e.
  \end{equation}

  The total Hamiltonian can then be expressed as:

  \begin{equation}
    \Ha = \Ha_n + \Ha_e = \Ha_{nn} + \Ha_{ne} + \Ha_{ee}.
  \end{equation}

  Here, the $n$-$n$ interactions are treated classically. The electronic
  Schrödinger equation thus takes the form:

  \begin{equation}
    \Ha_e \ket{\psi_e} = E_e \ket{\psi_e}.
  \end{equation}

  As a result, the total energy can be written as:

  \begin{equation}
    E = E_e + E_{nn},
  \end{equation}

  where $E_{nn}$ corresponds to the classical Coulomb repulsion between nuclei.
  At this point, we ``\textit{just}'' need to solve the electronic Schrödinger
  equation. The electronic Hamiltonian includes the kinetic energy of the
  electrons, the electron-nuclear attractions, and the electron-electron
  repulsions:

  \begin{equation}
    \Ha_e = -\sum_i \sfrac{1}{2}\nabla_i^2 - \sum_{i,\alpha}\frac{Z_\alpha}{|\mathbf{R}_\alpha - \mathbf{r}_i|}
      + \sum_{i>j} \frac1{|\mathbf{r}_i - \mathbf{r}_j|},
  \end{equation}

  where the $\alpha$ index runs over nuclei, and the $i,j$ indices run over
  electrons. The Hamiltonian is expressed in atomic units, i.e.\ $\hbar = m_e = e = 1$.

}
