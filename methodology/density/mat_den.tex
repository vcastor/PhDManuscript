% -*- coding: utf-8 -*-
\newpage
\subsection{Density matrix}

The density matrix offers a compact and versatile way to encode information
about a quantum system, particularly when working with reduced representations
of many-electron wavefunctions~\cite{Lwdin1955, McWeeny1960}. Rather than
relying on the full $N$-electron wavefunction $\Psi$, many physical properties
can be derived from the reduced density matrices, which are central to modern
electronic structure theory.

In quantum mechanics, the expectation value of an observable associated with an
operator $\Qop$ for a system described by $\Psi$ is given by:

\begin{align}
  \langle \Qop \rangle = \int\Psi^{\star}(\mathbf{x}_1,\dots,\mathbf{x}_n) \Qop \Psi(\mathbf{x}_1,\dots,\mathbf{x}_n)
    \dd\mathbf{x}_1 \dots \dd\mathbf{x}_n.
\end{align}

When the operator explicitly depends on only $m$ variables ($m \leq n$), the
expectation value can be simplified to:

\scriptsize
\begin{align}
  \langle \Qop \rangle & = \int\Qop\Psi
    (\mathbf{x}_1,\ldots , \mathbf{x}_m, \mathbf{x}_{m+1}, \ldots , \mathbf{x}_n)
    \Psi^{\star} (\mathbf{x}_1,\ldots ,\mathbf{x}_m, \mathbf{x}_{m+1}, \ldots , \mathbf{x}_n)
    \dd\mathbf{x}_1\ldots \dd\mathbf{x}_n \nonumber \\
	& = \int\left(\int
    \Psi^{\star}(\mathbf{x}_1,\ldots , \mathbf{x}_m, \mathbf{x}_{m+1}, \ldots , \mathbf{x}_n)
    \dd \mathbf{x}_{m+1}\ldots \dd \mathbf{x}_n \right)
    \Qop \Psi(\mathbf{x}_1,\ldots , \mathbf{x}_m, \mathbf{x}_{m+1}, \ldots , \mathbf{x}_n)
    \dd\mathbf{x}_1\ldots \dd\mathbf{x}_m \nonumber\\
	& =\int
    \Qop (\mathbf{x}_1, \ldots , \mathbf{x}_m) F_m
    (\mathbf{x}_1, \ldots , \mathbf{x}_m)
    \dd\mathbf{x}_1\ldots\dd\mathbf{x}_m,
\end{align}
\normalsize

\noindent where the function $F_m$ is defined as the integral over the remaining variables:
%
\footnotesize
\begin{align}
  F_m(\mathbf{x}_1, \ldots , \mathbf{x}_m) =
  \int\Psi(\mathbf{x}_1,\ldots , \mathbf{x}_m, \mathbf{x}_{m+1}, \ldots , \mathbf{x}_n)
    \Psi^{\star}(\mathbf{x}_1,\ldots , \mathbf{x}_m, \mathbf{x}_{m+1}, \ldots , \mathbf{x}_n)
    \dd \mathbf{x}_{m+1}\ldots \dd \mathbf{x}_n .
\end{align}
\normalsize

% The function $F_m$ is directly related to the $m$-th order density matrix
% $\Gamma_m$, which provides a systematic approach to capturing correlations
% between electrons. The $m$-th order density matrix is defined as:

\noindent This object $F_m$ is identified with the $m$-th order reduced density matrix
$\Gamma_m$:

\footnotesize
\begin{align}
  \Gamma_m(\mathbf{x}_{1}, \ldots , \mathbf{x}_m; \mathbf{x}^\prime_1, \ldots , \mathbf{x}^\prime_m)
  = {n \choose m} \int
    \Psi^{\star}(\mathbf{x}^\prime_1,\ldots , \mathbf{x}^{\prime}_n)
    \Psi(\mathbf{x}_1,\ldots ,\mathbf{x}_n)
    \dd \mathbf{x}_{m+1}\ldots \dd \mathbf{x}_n .
\end{align}
\normalsize

\newpage
Reduced density matrices form a hierarchy, as higher-order matrices can be
contracted into lower-order ones:

\begin{align}
  \Gamma_m(\mathbf{x}_1, \ldots , \mathbf{x}_m; \mathbf{x}'_1, \ldots , \mathbf{x}'_m)
  = \frac{m+1}{n-m} \int \Gamma_{m+1}(\mathbf{1}, \ldots , \mathbf{m+1}; \mathbf{1}', \ldots , \mathbf{m+1}') \dd\mathbf{x}_{m+1}.
\end{align}

In quantum chemistry, the operators of primary interest are typically mono- or
bi-electronic, making the first- and second-order density matrices particularly
relevant:

\begin{gather}
  \label{m_primer_orden}
  \Gamma_1 (\mathbf{x}_1;\mathbf{x}_1^{\prime}) =
    n\int\Psi^\star(\mathbf{1}^\prime, \mathbf{2}^\prime, \ldots ,\mathbf{n}^\prime)
    \Psi(\mathbf{1}, \mathbf{2}, \ldots , \mathbf{n})
    \dd \mathbf{x}_2 \ldots \dd \mathbf{x}_n, \\
  \Gamma_2 (\mathbf{x}_1, \mathbf{x}_2;\mathbf{x}^\prime_1,\mathbf{x}^\prime_2)
    = n(n-1) \int\Psi^\star(\mathbf{1}^\prime, \mathbf{2}^\prime, \ldots ,\mathbf{n}^\prime)
    \Psi(\mathbf{1}, \mathbf{2}, \ldots , \mathbf{n}) \dd \mathbf{x}_3 \ldots \dd \mathbf{x}_n.
\end{gather}

The first-order density matrix (Equation~\ref{m_primer_orden}), often called the
Fock-Dirac density matrix, becomes especially useful upon integration over spin
coordinates, yielding the first-order reduced density matrix:

\begin{align}
  \rho_1(\mathbf{r}_1,\mathbf{r}^\prime_1) =
    \int\Gamma_1 (\mathbf{x}_1;\mathbf{x}_1^{\prime}) \dd \omega_1.
  \label{m_r_1}
\end{align}

Unlike density functions, density matrix elements generally lack direct
physical interpretation, with the notable exception of the diagonal elements of
the first-order reduced density matrix, which represent the electronic density
itself. Integrating these diagonal elements over all space yields the total
number of electrons in the system, highlighting their importance. Consequently,
many crucial properties ---particularly energies--- are conveniently expressed
using the first-order reduced density matrix and the electron pair
density~\cite{helgaker}.

\vfill
