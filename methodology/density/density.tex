% -*- coding: utf-8 -*-
\newpage
\section{Electronic Density}\label{densidades}

The electronic density, $\rho(\mathbf{r})$, is a fundamental concept in modern
theoretical chemistry. It offers a compact yet informative description of a
system's electronic structure and plays a central role in density-based methods
such as Density Functional Theory (DFT), as well as in interpretative
frameworks like QTAIM.

For a system comprising two electrons with spin-spatial coordinates
$\mathbf{x}_1$ and $\mathbf{x}_2$, the electronic density is given by
$|\Psi(\mathbf{x}_1, \mathbf{x}_2)|^2$~\cite{RobertG1994}.  To calculate the
probability of finding electron 1 in the volume element $\dd\mathbf{x}_1$ and
electron 2 in $\dd\mathbf{x}_2$, the full probability density must be
integrated over $\dd\omega_1 \dd\mathbf{x}_2$, where $\dd\mathbf{x}_n =
\dd\tau_n \dd\omega_n$.

\nt{

  Since the electrons are indistinguishable,
  for a system with $N$ electrons, the electronic density can be
  generalised as:
  %
  \begin{align}
    \rho(\mathbf{r})=N\int |\Psi (\mathbf{x}_1 ,  \mathbf{x}_2,  \ldots , \mathbf{x}_N)|^2
    \dd\omega_1 \dd\mathbf{x}_2 \ldots \dd\mathbf{x}_N ,
    \label{rho_definition}
  \end{align}

  \noindent such that:
  \begin{align}
    \int\rho (\mathbf{r})\dd\tau = N .
  \end{align}

}

In the case of a Slater Determinant (Hartree-Fock), the electronic density for
a closed shell with spatial orbitals ${\psi_a}$ can be written as:
%
\begin{align}
  \rho (\mathbf{r}) = 2 \sum_{a=1}^{\sfrac{N}{2}} |\psi_a (\mathbf{r})|^2 .
\end{align}

\newpage
For a system with $N$ particles, it is possible to write an operator for the
density $\hat{\rho}$ and then, get the expected value for the system
wavefunction,
%
\begin{align}
  \hat{\rho} = \sum_i^N \hat{\delta} (\mathbf{r}_i - \mathbf{r}_0),
\end{align}

\noindent then $\rho (\mathbf{r})$ is a expected value of a quantum mechanics
operator, which can be written as:
%
\begin{align}
  \rho (\mathbf{r}_0) &= \int\Psi^{\star} (\mathbf{x}_1, \ldots , \mathbf{x}_N)
  \sum_i^N \hat{\delta} (\mathbf{r}_i - \mathbf{r}_0) \Psi (\mathbf{x}_1, \ldots , \mathbf{x}_N)
  \dd\mathbf{x}_1 \ldots \dd\mathbf{x}_N \nonumber \\
  & = \bra{\Psi} \hat{\rho} \ket{\Psi} 
\end{align}

The electronic density is not only central to theory but also accessible
experimentally. Unlike many aspects of the electronic structure,
$\rho(\mathbf{r})$ can be directly measured using techniques such as X-ray and
neutron diffraction~\cite{Kasai2018, Coppens1971}. This makes it a valuable
bridge between theory and experiment, allowing direct comparison between
computed and observed densities, a stringent test of the accuracy and
reliability of computational methods.

Going forward, it is also possible to establish a density function for an electron
couple, also called pair density, such that:
%
\begin{align}
  \rho_2(\mathbf{r}_1,\mathbf{r}_2) = N(N-1)\int |\Psi(\mathbf{x}_1, \mathbf{x}_2, \ldots , \mathbf{x}_N)|^2
  \dd\omega_1 \dd\omega_2  \dd\mathbf{x}_3 \ldots \dd\mathbf{x}_N ,
  \label{d_pares}
\end{align}

\noindent where $\rho_2(\mathbf{r}_1,\mathbf{r}_2)N^{-1}(N-1)^{-1}$ determines
the normalised probability of simultaneously finding two electrons per volume
centered at positions $\mathbf{r}_1$ and $\mathbf{r}_2$.  Two normalisation
conventions are found in the literature: Löwdin~\cite{Lwdin1955} introduced the
factor $N(N-1)$, corresponding to ordered electron pairs, while
McWeeny~\cite{mcweeny1969methods} proposed an additional division by two to
account for unordered pairs. This pair density is commonly expressed as a sum
of an uncorrelated term and a correlation contribution, $\rho_2(\mathbf{r}_1,
\mathbf{r}_2) = \rho(\mathbf{r}_1)\rho(\mathbf{r}_2) +
\rho_2^{xc}(\mathbf{r}_1, \mathbf{r}_2)$.

The importance of the one- and two-electro densities stems from the fact that,
within the \gls{BOA} and in the absence of external fields, the total
electronic energy of the system can be expressed entirely in terms of these
densities.
%
\keq{}{

  \begin{align}
    E = &-\frac12\int\nabla^2\rho_1(\mathbf{r}_1,
      \mathbf{r}^{\,\prime}_1) \biggr |_{\mathbf{r}^{\,\prime}_1 \rightarrow \mathbf{r}_1}
      \dd\tau_1 -\sum_A\int\frac{Z_A \rho_1 (\mathbf{r}_1)}{r_{1A}}\dd\tau_1
      +\sum_{A\neq B}\frac{Z_A Z_B}{r_{AB}} \nonumber \\
    \phantom{=} & +\frac12\int\int\frac{\rho (\mathbf{r}_1) \rho
      (\mathbf{r}_2)}{r_{12}} \dd\tau_1 \dd\tau_2
      +\frac12\int\int\frac{\rho_2^{xc}(\mathbf{r}_1, \mathbf{r}_2)}{r_{12}} \dd\tau_1
      \dd\tau_2 \nonumber \\
    =&\ T + V_{ne} + V_{nn} + V_{ee} + V_{xc} ,
  \label{E_b-o}
  \end{align}

  the first term represents the kinetic energy of the electrons, while the
  second and third correspond to the electron-nucleus and nucleus-nucleus
  interactions, respectively. The fourth term describes the classical
  electron-electron repulsion, treated as the Coulomb interaction between
  one-electron densities. The final term introduces exchange-correlation effects
  through the pair density.

}

The exchange-correlation term, $V_{xc}$, is introduced by subtracting the
uncorrelated product $\rho(\mathbf{r}_1)\rho(\mathbf{r}_2)$ from the full pair
density $\rho_2(\mathbf{r}_1, \mathbf{r}_2)$. This isolates the
exchange-correlation contribution, capturing both the effects of electron
indistinguishability and the correlated response of the electron distribution
to the presence of another electron.

In practice, evaluating Equation~\ref{E_b-o} within an electronic structure
method requires expressing the electron densities in terms of the molecular
orbital basis functions $\phi_i$,
%
\begin{gather}
  \rho_1 (\mathbf{r}) = \sum_{ij} D_{ij}\phi_{i}(\mathbf{r})\phi_{j}(\mathbf{r}), \\
  \rho_2 (\mathbf{r}_1, \mathbf{r}_2) = \sum_{ijkl}d_{ijkl}\phi_{i}(\mathbf{r}_1) \phi_{j}(\mathbf{r}_1)
    \phi_{k}(\mathbf{r}_2)\phi_{l}(\mathbf{r}_2),
\end{gather}

\noindent where $D_{ij}$ and $d_{ijkl}$ are the elements of the first-
and second-order density matrices, respectively.

% --------------------------
% Subsection: Density Matrix
% -*- coding: utf-8 -*-
\newpage
\subsection{Density matrix}

The density matrix offers a compact and versatile way to encode information
about a quantum system, particularly when working with reduced representations
of many-electron wavefunctions~\cite{Lwdin1955, McWeeny1960}. Rather than
relying on the full $N$-electron wavefunction $\Psi$, many physical properties
can be derived from the reduced density matrices, which are central to modern
electronic structure theory.

In quantum mechanics, the expectation value of an observable associated with an
operator $\Qop$ for a system described by $\Psi$ is given by:

\begin{align}
  \langle \Qop \rangle = \int\Psi^{\star}(\mathbf{x}_1,\dots,\mathbf{x}_n) \Qop \Psi(\mathbf{x}_1,\dots,\mathbf{x}_n)
    \dd\mathbf{x}_1 \dots \dd\mathbf{x}_n.
\end{align}

When the operator explicitly depends on only $m$ variables ($m \leq n$), the
expectation value can be simplified to:

\scriptsize
\begin{align}
  \langle \Qop \rangle & = \int\Qop\Psi
    (\mathbf{x}_1,\ldots , \mathbf{x}_m, \mathbf{x}_{m+1}, \ldots , \mathbf{x}_n)
    \Psi^{\star} (\mathbf{x}_1,\ldots ,\mathbf{x}_m, \mathbf{x}_{m+1}, \ldots , \mathbf{x}_n)
    \dd\mathbf{x}_1\ldots \dd\mathbf{x}_n \nonumber \\
	& = \int\left(\int
    \Psi^{\star}(\mathbf{x}_1,\ldots , \mathbf{x}_m, \mathbf{x}_{m+1}, \ldots , \mathbf{x}_n)
    \dd \mathbf{x}_{m+1}\ldots \dd \mathbf{x}_n \right)
    \Qop \Psi(\mathbf{x}_1,\ldots , \mathbf{x}_m, \mathbf{x}_{m+1}, \ldots , \mathbf{x}_n)
    \dd\mathbf{x}_1\ldots \dd\mathbf{x}_m \nonumber\\
	& =\int
    \Qop (\mathbf{x}_1, \ldots , \mathbf{x}_m) F_m
    (\mathbf{x}_1, \ldots , \mathbf{x}_m)
    \dd\mathbf{x}_1\ldots\dd\mathbf{x}_m,
\end{align}
\normalsize

\noindent where the function $F_m$ is defined as the integral over the remaining variables:
%
\footnotesize
\begin{align}
  F_m(\mathbf{x}_1, \ldots , \mathbf{x}_m) =
  \int\Psi(\mathbf{x}_1,\ldots , \mathbf{x}_m, \mathbf{x}_{m+1}, \ldots , \mathbf{x}_n)
    \Psi^{\star}(\mathbf{x}_1,\ldots , \mathbf{x}_m, \mathbf{x}_{m+1}, \ldots , \mathbf{x}_n)
    \dd \mathbf{x}_{m+1}\ldots \dd \mathbf{x}_n .
\end{align}
\normalsize

% The function $F_m$ is directly related to the $m$-th order density matrix
% $\Gamma_m$, which provides a systematic approach to capturing correlations
% between electrons. The $m$-th order density matrix is defined as:

\noindent This object $F_m$ is identified with the $m$-th order reduced density matrix
$\Gamma_m$:

\footnotesize
\begin{align}
  \Gamma_m(\mathbf{x}_{1}, \ldots , \mathbf{x}_m; \mathbf{x}^\prime_1, \ldots , \mathbf{x}^\prime_m)
  = {n \choose m} \int
    \Psi^{\star}(\mathbf{x}^\prime_1,\ldots , \mathbf{x}^{\prime}_n)
    \Psi(\mathbf{x}_1,\ldots ,\mathbf{x}_n)
    \dd \mathbf{x}_{m+1}\ldots \dd \mathbf{x}_n .
\end{align}
\normalsize

\newpage
Reduced density matrices form a hierarchy, as higher-order matrices can be
contracted into lower-order ones:

\begin{align}
  \Gamma_m(\mathbf{x}_1, \ldots , \mathbf{x}_m; \mathbf{x}'_1, \ldots , \mathbf{x}'_m)
  = \frac{m+1}{n-m} \int \Gamma_{m+1}(\mathbf{1}, \ldots , \mathbf{m+1}; \mathbf{1}', \ldots , \mathbf{m+1}') \dd\mathbf{x}_{m+1}.
\end{align}

In quantum chemistry, the operators of primary interest are typically mono- or
bi-electronic, making the first- and second-order density matrices particularly
relevant:

\begin{gather}
  \label{m_primer_orden}
  \Gamma_1 (\mathbf{x}_1;\mathbf{x}_1^{\prime}) =
    n\int\Psi^\star(\mathbf{1}^\prime, \mathbf{2}^\prime, \ldots ,\mathbf{n}^\prime)
    \Psi(\mathbf{1}, \mathbf{2}, \ldots , \mathbf{n})
    \dd \mathbf{x}_2 \ldots \dd \mathbf{x}_n, \\
  \Gamma_2 (\mathbf{x}_1, \mathbf{x}_2;\mathbf{x}^\prime_1,\mathbf{x}^\prime_2)
    = n(n-1) \int\Psi^\star(\mathbf{1}^\prime, \mathbf{2}^\prime, \ldots ,\mathbf{n}^\prime)
    \Psi(\mathbf{1}, \mathbf{2}, \ldots , \mathbf{n}) \dd \mathbf{x}_3 \ldots \dd \mathbf{x}_n.
\end{gather}

The first-order density matrix (Equation~\ref{m_primer_orden}), often called the
Fock-Dirac density matrix, becomes especially useful upon integration over spin
coordinates, yielding the first-order reduced density matrix:

\begin{align}
  \rho_1(\mathbf{r}_1,\mathbf{r}^\prime_1) =
    \int\Gamma_1 (\mathbf{x}_1;\mathbf{x}_1^{\prime}) \dd \omega_1.
  \label{m_r_1}
\end{align}

Unlike density functions, density matrix elements generally lack direct
physical interpretation, with the notable exception of the diagonal elements of
the first-order reduced density matrix, which represent the electronic density
itself. Integrating these diagonal elements over all space yields the total
number of electrons in the system, highlighting their importance. Consequently,
many crucial properties ---particularly energies--- are conveniently expressed
using the first-order reduced density matrix and the electron pair
density~\cite{helgaker}.

\vfill


