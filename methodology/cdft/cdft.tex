% -*- coding: utf-8 -*-
\newpage
\section{Conceptual DFT}

The topological analysis of the electron density in QTAIM represents just one 
facet of density-based approaches to chemical reactivity. A complementary 
perspective is offered by Conceptual Density Functional Theory (\gls{CDFT}), 
which focuses on how variations in global and local electronic descriptors 
govern chemical behaviour. The foundations of \gls{CDFT} can be traced to the 
work of Parr and co-workers, who identified the Lagrange multiplier 
in density functional theory ---traditionally associated with the chemical 
potential--- as the negative of electronegativity~\cite{Parr1978}.

\vspace{1em}%
This idea was further developed by Parr and Pearson in 1983, who related the
second derivative of the energy with respect to the number of electrons to the
concept of chemical hardness~\cite{Parr1983}. The formal introduction of the
Fukui function in 1984 extended the framework to local reactivity indices,
enabling site-specific analyses of chemical reactivity~\cite{Parr1984}. These
developments laid the groundwork for a density-based interpretation of chemical
concepts, with the term ``Conceptual DFT'' formally introduced in
1995~\cite{Parr1995}.

\vspace{1em}%
Conceptual DFT offers a hierarchy of global and local descriptors ---such as
the chemical potential, chemical hardness, softness, Fukui functions, dual
descriptors, and electrophilicity index--- which allow us to predict and
rationalise patterns of chemical reactivity.

\newpage
\subsection{Global Descriptors}

The cornerstone of this approach is the identification of the chemical
potential $\mu$ as the Lagrange multiplier associated with the constraint on
electron number in the Euler-Lagrange variational formalism:

\follow{Chemical Potential as Lagrange Multiplier}{

  The existence of the energy as a functional of the electron density,
  the constraint on the number of electrons, and the variational principle let
  us write the Euler-Lagrange equation as:
%
  \begin{align}
    \fdv{}{\rho(\mathbf{r})} \left( E[\rho] - \mu \left[
      \int \rho(\mathbf{r})\dd\mathbf{r} - N \right] \right) = 0
  \end{align}

  implying $\mu = \fdv{E[\rho]}{\rho(\mathbf{r})}$ as the Lagrange multiplier
  to enforce the electron-number constraint.

  % \begin{align}
  %   \left(\pdv{E}{N}\right)_{v(\mathbf{r})} & =
  %     \int \left(\fdv{E}{\rho}\right)_{v(\mathbf{r})}
  %       \left(\pdv{\rho}{N}\right)_{v(\mathbf{r})} \dd\mathbf{r} \\
  %     & =\mu \pdv{N} \left[\int \rho\dd\mathbf{r}\right] = \mu
  % \end{align}

}

Solving it for the exact density, the chemical potential can be expressed as
the partial derivative of the energy with respect to the number of electrons,
quantifing the change in ground-state energy upon an infinitesimal change in
electron number at constant external potential $v(\mathbf{r})$:
%
\begin{align}
  \mu = v(\mathbf{r}) + \fdv{F[\rho]}{\rho(\mathbf{r})}
  = \left(\pdv{E}{N}\right)_{v(\mathbf{r})}.
\end{align}

\nt{
  Using the chain rule, we can show that:
%
  \begin{align}
    \left(\pdv{E}{N}\right)_{v(\mathbf{r})} & =
      \int \left(\fdv{E}{\rho}\right)_{v(\mathbf{r})}
        \left(\pdv{\rho}{N}\right)_{v(\mathbf{r})} \dd\mathbf{r} \\
      & =\mu \pdv{N} \left[\int \rho\dd\mathbf{r}\right] = \mu.
  \end{align}

}

Accordingly, its sign and magnitude measure the system's propensity to exchange
electrons with its surroundings: a more negative $\mu$ indicates a stronger
tendency to accept electronic charge and vice versa.

% The Lagrange multiplier $\mu$ arises in the constrained minimisation of 
% the energy functional with a fixed number of electrons, is identified as the 
% chemical potential. This quantity describes the system's tendency to 
% gain or lose electrons, and can thus be interpreted as the escaping tendency 
% of an electron from the system.

\newpage
From a thermodynamic perspective, the chemical potential governs the direction
of charge transfer, favouring flow from regions of high $\mu$ to low $\mu$.
Importantly, $\mu$ can be connected to experimentally accessible quantities
such as the vertical ionisation potential $I$ and the vertical electron
affinity $A$.

\vspace{1em}%
Using a finite difference approximation under the assumption of piecewise
linearity of the energy with respect to electron number, we obtain:
%
\begin{align}
  \mu = \left(\pdv{E}{N}\right)_{v(\mathbf{r})}
    \approx -\chi = \frac{I + A}2,
\end{align}

% \newpage
\noindent where $\chi$ is the electronegativity, traditionally defined as the
tendency of an atom or molecule to attract electrons. In this formulation,
electronegativity and chemical potential are essentially two sides of the same
coin, differing only by a sign.

\vspace{1em}%
Within the Koopmans' theorem, the chemical potential can be
approximated using frontier molecular orbital energies. Specifically, the
ionisation potential $I$ and electron affinity $A$ are associated with the
negative of the \homo and \lumo energies,
respectively~\cite{Pearson1986}.

\nt{
  Consequently, the chemical potential can be expressed as:
  \begin{align}
    \mu \approx -\chi = \nicefrac12 (\epsilon_\mathrm{HOMO} + \epsilon_\mathrm{LUMO}).
  \end{align}
}

\newpage
Building on the Hard and Soft Acids and Bases principle, Pearson introduced the
concept of chemical hardness $\eta$~\cite{Pearson1997}, which measures a
system's resistance to change in electron number. Formally, it corresponds to
the second derivative of the energy with respect to the number of
electrons~\cite{Parr1983}, and the inverse of the hardness
defines the softness $S$~\cite{Yang1985}, a measure of
the system's reactivity or its ability to adapt its electron density in
response to external perturbations:

\vspace{1em}%
\dfn{Softness and Hardness}{
  \begin{align}
    \eta & = \left(\pdv[2]{E}{N}\right)_{v(\mathbf{r})}
      = \left(\pdv{\mu}{N}\right)_{v(\mathbf{r})} \\
    & = I - A 
      \approx -(\epsilon_\mathrm{HOMO} - \epsilon_\mathrm{LUMO}),
  \end{align}
  \begin{align}
    S = \frac1\eta = \left(\pdv{N}{\mu}\right)_{v(\mathbf{r})}
  \end{align}
}

\vspace{1em}%
This global descriptor balances the stabilizing energy a system gains upon 
acquiring charge (via $\mu$) with its resistance to such charge transfer 
(via $\eta$), offering a compact measure of electrophilic power.

\vspace{1em}%
Hardness is generally associated with a system's resistance to deformation of
its electron density under external perturbations. As such, hard systems tend
to exhibit low polarisability and magnetisability, while soft systems respond
more readily to changes in their environment. To further quantify a system's
reactivity, Parr introduced the electrophilicity
index~\cite{Parr1999}, which combines the tendency of a system to acquire
electrons (through the chemical potential) with its resistance to charge
transfer (through the hardness):
%
\begin{align}
  \omega = \frac{\mu^2}{2\eta} = \frac{\chi^2}{2\eta}.
\end{align}

\newpage
\subsection{Local Descriptors}

Many methods have been developed to predict if a chemical reaction will happen
or not, and in case it happens, how it will proceed. One of the most popular
method of prediction is frontier molecular orbital (\gls{FMO})
theory~\cite{Albright2013}, which relies on the shapes and symmetries of the
highest occupied and lowest unoccupied molecular orbitals (\homo and
\lumo).

\vspace{1em}%
However, a major limitation of \gls{FMO} theory is its reliance on a fixed
orbital picture, which does not account for electron correlation or orbital
relaxation. This limitation motivated the development of the Fukui
function~\cite{Ayers2000, Yang1984, Parr1984} within the context of DFT a
reactivity descriptor that preserves the conceptual foundation of \gls{FMO}
while, in principle, incorporating both electron correlation~\cite{Melin2005}
and orbital relaxation effects~\cite{Yang1984, Bartolotti2005, Anderson2007}.

\vspace{1em}%
\dfn{Fukui Function}{
  \begin{equation}
    f(\mathbf{r}) = \left[\fdv{\mu}{v(\mathbf{r})}\right]_N
      = \left[\pdv{\rho(\mathbf{r})}{N}\right]_{v(\mathbf{r})}
  \end{equation}
  where $\mu$ is the electronic chemical potential, the negative of the electronegativity,
  $v(\mathbf{r})$ is the external potential due to the atomic nuclei, $\rho (\mathbf{r})$ is the
  electron density, and $N$ is the number of electrons.
}

\vspace{1em}%
Because the electron density $\rho(\mathbf{r})$ exhibits discontinuities as a
function of the number of electrons $N$~\cite{Perdew1982, Zhang2000}, its
derivative with respect to $N$ must be evaluated from the left and from the
right.

\newpage
This yields two distinct Fukui functions, each appropriate for describing
different types of reactivity: one for nucleophilic attack (electron addition)
and another for electrophilic attack (electron removal).

\follow{Fukui functions $f^\pm(\mathbf{r})$}{

  \begin{align}
    f^\pm(\mathbf{r}) = \left[\pdv{\rho(\mathbf{r})}{N}\right]^\pm_{v(\mathbf{r})} =
    \begin{cases}
      f^+(\mathbf{r}) &= \rho(N+1) - \rho(N) \\
      f^-(\mathbf{r}) &= \rho(N) - \rho(N-1)
    \end{cases}
  \end{align}

  If we consider the density as a function of the \gls{KS} orbitals and the
  occupation numbers,

  \begin{align}
    \rho(\mathbf{r}) = \sum_{i=1}^{\infty} n_i | \phi_i(\mathbf{r}) |^2 \quad \text{with} \quad
    n_i =
    \begin{cases}
        1 & i \leq HOMO \\
        0 & i \geq LUMO
    \end{cases}
  \end{align}

  \noindent then, the Fukui functions became:

  \begin{align}
    f^+ (\mathbf{r}) &= |\phi_{\mathrm{LUMO}}(\mathbf{r})|^2 + \sum_\mathrm{LUMO}^\infty
    \left( \pdv{|\phi_i (\mathbf{r})|^2}{N} \right) ^+_{v(\mathbf{r})} \\
    f^- (\mathbf{r}) &= |\phi_{\mathrm{HOMO}}(\mathbf{r})|^2 + \sum^\mathrm{HOMO}_1
    \left( \pdv{|\phi_i (r)|^2}{N} \right) ^-_{v(\mathbf{r})}
  \end{align}

  And therefore, neglecting the orbital relaxation terms, from linking
  to frontier molecular orbital theory,

  \begin{align}
    f^+ (\mathbf{r}) \approx | \phi_{\mathrm{LUMO}} |^2 \\
    f^- (\mathbf{r}) \approx | \phi_{\mathrm{HOMO}} |^2
  \end{align}

}

Additionally, the dual descriptor, proposed by Morell~\cite{Morell2004}, is
defined as the mixed second derivative of the energy with respect to the
external potential and the number of electrons, and equivalently, as the second
derivative of the electron density with respect to the number of electrons:
%
\begin{align}
  f^{(2)}(\mathbf{r}) = \frac{\delta^3E}{\partial N^2\delta v(\mathbf{r})} =
    \left(\pdv[2]{\rho(\mathbf{r})}{N}\right)_{v(\mathbf{r})} =
    \left(\pdv{f(\mathbf{r})}{N}\right)_{v(\mathbf{r})} =
    \left(\fdv{\eta}{v(\mathbf{r})}\right)_N
\end{align}

\newpage
In practice, the dual descriptor is computed as the difference between the
electrophilic and nucleophilic Fukui functions at a given point $\mathbf{r}$. Its
sign provides insight into local reactivity: if $f^{(2)}(\mathbf{r}) > 0$, the
site is more electrophilic than nucleophilic; conversely, if $f^{(2)}(\mathbf{r})
< 0$, the site is more nucleophilic than electrophilic.

The linear response function $\chi(\mathbf{r}, \mathbf{r}')$, ---sometimes
denoted as $\omega(\mathbf{r}, \mathbf{r}')$ and named polarisability
kernel---~\cite{Langenaeker2001} has been employed to probe local and non-local
features of chemical reactivity. This function quantifies how the electron
density at a point $\mathbf{r}$ responds to an external perturbation applied at
another point $\mathbf{r}'$. Formally, it is defined as~\cite{Gzquez2021}:

\begin{align}
  \chi(\mathbf{r},\mathbf{r}') &=
    \left(\fdv{\rho(\mathbf{r})}{v(\mathbf{r}')}\right)_{N}
    % = \sum_{k,l}^K q_{kl} \beta_k (\mathbf{r})\beta_l (\mathbf{r}')
    = \sum_{k}^\infty q_{j} \beta_k (\mathbf{r})\beta_k (\mathbf{r}').
  % \chi_{\mathrm{AB}} &= \int{V_{\mathrm{A}}}\int{V_{\mathrm{B}}} \dd\mathbf{r} \dd\mathbf{r}' \chi(\mathbf{r},\mathbf{r}') \\
  % &=\sum_{k\in \mathrm{A}} \sum_{l\in \mathrm{B}} q_{kl} \int \dd\mathbf{r}\beta_k(\mathbf{r}) \int \dd\mathbf{r}'\beta_l(\mathbf{r}')
\end{align}

This descriptor provides insight into a variety of chemically meaningful
phenomena, including electron delocalisation, inductive and mesomeric effects,
as well as aromaticity, thereby offering a more fundamental basis for
understanding chemical reactivity~\cite{Geerlings2003}.
% And can be
% sumarised in the~\Figure{fig_canonical}.

The work of Langenaeker and Liu~\cite{Langenaeker2001} was instrumental in this context, as
they analysed the response of the electron density to perturbations in the
external potential for atomic systems. Their study highlighted the potential of
the linear response function to reveal the intrinsic reactivity patterns of
atoms and molecules, going beyond orbital-based or purely local descriptors.

In the same way that global hardness provides insight into which species is
more likely to undergo a chemical reaction, local hardness~\cite{Yang1985}
indicates the most reactive site within a molecule. It is defined as the
functional derivative of the chemical potential with respect to the electron
density:

\begin{align}
  \eta(\mathbf{r}) =
    \left(\fdv{\mu}{\rho(\mathbf{r})}\right)_{v(\mathbf{r})}.
\end{align}

\pagebreak
Several definitions for local hardness have been proposed. Notably, the
expressions introduced by Meneses~\cite{Meneses2004}
($\eta_\mathrm{M}(\mathbf{r})$) and by Gál~\cite{Gal2011}
($\eta_\mathrm{G}(\mathbf{r})$) are among the most widely used:

\begin{gather}
  \eta_\mathrm{M}(\mathbf{r}) =
    \left(\frac{f^+(\mathbf{r}) - f^-(\mathbf{r})}{2}\right) +
      \mu f^{(2)}(\mathbf{r}) \\
  \eta_\mathrm{G}(\mathbf{r}) =
    If^-(\mathbf{r}) + Af^+(\mathbf{r})
\end{gather}

Similarly, a local softness can be defined, though it is not merely the inverse
of local hardness. It is more appropriately expressed as the product of the
global softness and the Fukui functions:

\begin{align}
  s^\pm(\mathbf{r}) = Sf^\pm(\mathbf{r}).
\end{align}

By integrating the local softness over the entire space, one recovers the global
softness, since the Fukui functions integrate to unity. At the scale of a single
molecule, local softness conveys information similar to that provided by the
Fukui functions. However, its principal advantage lies in the possibility of
comparing reactive sites across different molecules, something the Fukui function
alone cannot achieve.

It is important to remark that the previous discussion is based on the
canonical ensemble, however, in a context of a finite temperature, the
functional $\Omega$, the grand potetial, is defined $\Omega = E - N\mu$ or
$\Omega = E - \mu (N - N_0)$, where $N_0$ is the number of electrons in the
reference state. At given temperature $T$, the following hierarchy of response
functions was done~\cite{Chermette1999}. In the Figure~\ref{canetgrand} we can
see the energy derivatives and the comparison between the canonical and grand
canonical ensembles.

\newpage
\vspace*{8em}%
\begin{figure}[h!]
  \centering
  \begin{subfigure}[t]{1\textwidth}
    \centering
    \begin{tikzpicture}[
    var/.style={draw, rounded corners=2pt, minimum width=3.5cm, minimum height=1cm, fill=blue!5},
    deriv/.style={draw=none, fill=none, font=\footnotesize\itshape},
    >={Stealth[round]}, thick
  ]

  % Level 0 - centered
  \node[var] (E) at (0,0) {$E = E[N, v(\mathbf{r})]$};

  % Level 1 - centered symmetrically under E
  \node[var] (mu) at (-2.5,-2) {$\mu$};
  \node[var] (rho) at (2.5,-2) {$\rho(\mathbf{r})$};

  % Level 2 - centered with equal spacing
  \node[var] (eta) at (-5,-4) {$\eta$};
  \node[var] (f) at (0,-4) {$f(\mathbf{r})$};
  \node[var] (chi) at (5,-4) {$\chi(\mathbf{r}, \mathbf{r'})$};

  % Derivative labels level 1
  \node[deriv] at ($(E)!0.5!(mu)$) [xshift=-1.5em] 
      {$\partial_N$};
  \node[deriv] at ($(E)!0.5!(rho)$) [xshift=1.5em] 
      {$\delta_{v(\mathbf{r})}$};
    
  % Derivative labels level 2
  \node[deriv] at ($(mu)!0.5!(eta)$) [xshift=-1.5em] 
      {$\partial_N$};
  \node[deriv] at ($(mu)!0.5!(f)$) [xshift=1.5em] 
      {$\delta_{v(\mathbf{r})}$};
  \node[deriv] at ($(rho)!0.5!(f)$) [xshift=-1.5em] 
      {$\partial_N$};
  \node[deriv] at ($(rho)!0.5!(chi)$) [xshift=1.5em] 
      {$\delta_{v(\mathbf{r})}$};

  % Arrows
  \draw[->] (E) -- (mu);
  \draw[->] (E) -- (rho);
  \draw[->] (mu) -- (eta);
  \draw[->] (mu) -- (f);
  \draw[->] (rho) -- (f);
  \draw[->] (rho) -- (chi);

\end{tikzpicture}


    \caption{Energy derivatives and response functions in the Canonical Ensamble.}
    \label{fig_canonical}
  \end{subfigure}

  \vspace{4em}%

  \begin{subfigure}[t]{1\textwidth}
    \centering
    \begin{tikzpicture}[
    var/.style={draw, rounded corners=2pt, minimum width=3.5cm, minimum height=1cm, fill=blue!5},
    deriv/.style={draw=none, fill=none, font=\footnotesize\itshape},
    >={Stealth[round]}, thick
  ]

  % Level 0 - centered
  \node[var] (Omega) at (0,0) {$\Omega = \Omega[\mu, v(\mathbf{r})]$};

  % Level 1 - centered symmetrically under Omega
  \node[var] (N) at (-2.5,-2) {$-N$};
  \node[var] (rho) at (2.5,-2) {$\rho(\mathbf{r})$};

  % Level 2 - centered with equal spacing
  \node[var] (S) at (-5,-4) {$S$};
  \node[var] (s) at (0,-4) {$s(\mathbf{r})$};
  \node[var] (srr) at (5,-4) {$s(\mathbf{r}, \mathbf{r'})$};

  % Derivative labels level 1
  \node[deriv] at ($(Omega)!0.5!(N)$) [xshift=-1.5em] 
      {$\partial_\mu$};
  \node[deriv] at ($(Omega)!0.5!(rho)$) [xshift=1.5em] 
      {$\delta_{v(\mathbf{r})}$};
    
  % Derivative labels level 2
  \node[deriv] at ($(N)!0.5!(S)$) [xshift=-1.5em]
      {$\partial_\mu$};
  \node[deriv] at ($(N)!0.5!(s)$) [xshift=1.5em] 
      {$\delta_{v(\mathbf{r})}$};
  \node[deriv] at ($(rho)!0.5!(s)$) [xshift=-1.5em] 
      {$\partial_N$};
  \node[deriv] at ($(rho)!0.5!(srr)$) [xshift=1.5em] 
      {$\delta_{v(\mathbf{r})}$};

  % Arrows
  \draw[->] (Omega) -- (N);
  \draw[->] (Omega) -- (rho);
  \draw[->] (N) -- (S);
  \draw[->] (N) -- (s);
  \draw[->] (rho) -- (s);
  \draw[->] (rho) -- (srr);

\end{tikzpicture}


    \caption{Energy derivatives and response functions in the Grand Canonical Ensemble.}
    \label{fig_grandcanonical}
  \end{subfigure}
  \caption{Comparison of energy derivatives n the
    Canonical and Grand Canonical ensembles. The arrows indicate with
    respect variable the derivative is taken.}
  \label{canetgrand}
\end{figure}
\vfill%

