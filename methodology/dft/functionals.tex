% -*- coding: utf-8 -*-
\subsubsection{Functionals in this work}

This thesis primarily employs three exchange-correlation functionals:
PBE~\cite{Perdew1996}, PBE0~\cite{Adamo1999}, and M06-2X~\cite{Zhao2007}. These
were chosen for their proven balance between accuracy and computational
efficiency, and their suitability for describing geometries, reaction
energetics, and other electronic properties relevant to the systems studied.

\newpage
\noindent \textbf{PBE and PBE0}

\begin{sloppypar}
\noindent The Perdew-Burke-Ernzerhof (PBE) functional is a widely used
\gls{GGA} in \gls{DFT}~\cite{Perdew1996}. It was developed as a simplified and
improved alternative to the earlier PW91 functional~\cite{burke1998derivation},
which, despite its accuracy, exhibited some limietations: excessive complexity,
non-transparent analytic forms, over-parameterisation, discontinuous parameter
transitions, and incorrect scaling behaviour in certain density limits.
\end{sloppypar}

The correlation energy in PBE can be expressed as a sum of the uniform
electron-gas correlation energy, $\varepsilon_C^{\mathrm{unif}}$, and a
gradient correction term, $H$:
%
\begin{align}
  E_C^{\mathrm{PBE}}[\rho_{\uparrow}, \rho_{\downarrow}] =
    \int n(r)\left[\varepsilon_C^{\mathrm{unif}}(r_s,\zeta) + H(r_s,\zeta,t)\right]\dd\tau,
\end{align}

\vspace{-0.7em}%
\noindent where $r_s$ is the local Seitz radius, $\zeta$ the spin polarisation,
and $t$ a reduced density gradient. The gradient correction $H$ is designed
to satisfy conditions at slowly and rapidly varying density limits, as well as
to avoid logarithmic singularities:
%
\begin{align}
  H = \sfrac{e^2}{a_0}\gamma\phi^3 \ln
    \left(1 + \frac\beta\gamma t^2 \frac{1+At^2}{1+At^2+A^2t^4}\right).
\end{align}

\vspace{-0.7em}%
The exchange energy functional in PBE, meanwhile, is constructed to fulfil
conditions such as correct scaling behaviour, exact spin-scaling relations, and
compliance with the Lieb-Oxford bound~\cite{Lieb1979, Lieb1981}. Its
enhancement factor $F_X(s)$ is given by:
%
\begin{align}
  F_X(s) = 1 + \kappa - \frac\kappa{1 + \sfrac{\mu s^2}{\kappa}}.
\end{align}

\vspace{-0.7em}%
\noindent where $F_X(s)$ is the (dimensionless) exchange enhancement factor
that multiplies the \gls{LDA} exchange energy density, and $s$ is another
reduced density gradient,
%
\begin{align}
  s = \frac{|\nnabla\rho|}{2(3\pi^2)^{\sfrac{1}{3}}\rho^{\sfrac{4}{3}}}.
\end{align}

\vspace{-0.7em}%
The hybrid variant PBE0 introduces Hartree-Fock exact exchange into the PBE
formulation, improving predictions of electronic structure properties:
%
\begin{align}
  E_{XC}^{\mathrm{PBE0}} = a_0E_X^{\HF} + (1-a_0)E_X^{\mathrm{PBE}} + E_C^{\mathrm{PBE}},
\end{align}

\noindent typically setting $a_0 = 0.25$ (25 \% exact exchange).

\newpage
\noindent \textbf{M06-2X}

\noindent The M06-2X functional~\cite{Zhao2007} is a hybrid meta-GGA designed
specifically to provide improved accuracy for non-covalent interactions,
thermochemistry, and barrier heights. It incorporates 54 \% exact
\gls{HF} exchange and includes additional dependence on the kinetic energy
density, making it particularly effective in describing systems with pronounced
electron delocalisation or dispersion interactions.

The meta-GGA formulation of M06-2X involves three primary density-dependent
variables: the spin densities ($\rho_\sigma$), the reduced density gradients
($s_\sigma$), and the spin-dependent kinetic energy density ($\tau_\sigma$).

%\begin{align}
%  x_\sigma = \frac{|\nnabla \rho_\sigma|}{\rho^{\sfrac{4}{3}}_\sigma}.
%  %, \quad \tau_\sigma = \frac12\sum_{i}|\nabla \psi_{i,\sigma}(r)|^2.
%\end{align}

The functional construction involves expressions adapted from the VSXC
functional~\cite{VanVoorhis1998}, using parameters fitted to empirical data.
M06-2X defines auxiliary variables ($z_\sigma$) and functions ($\gamma$, $h$).

% \begin{align}
%   \begin{split}
%     z_\sigma &= \frac{2\tau_\sigma}{\rho_\sigma^{\sfrac53}} - C_F\\
%     C_F &= \frac35 \left( 6\pi^2 \right)^{\sfrac23}\\
%     \gamma(x_\sigma, z_\sigma ) &= 1 +\alpha (x_\sigma^2 +z_\sigma)\\
%     h(x_\sigma, z_\sigma ) &=
%       \frac{d_0}{\gamma(x_\sigma, z_\sigma )} +
%       \frac{d_1x\sigma^2 + d_2z_\sigma}{\gamma^2(x_\sigma, z_\sigma )} +
%       \frac{d_3x_\sigma^4 + d_4x_\sigma^2z_\sigma + d_5z_\sigma^2}{\gamma^3(x_\sigma, z_\sigma )}
%   \end{split}
%   \label{h_de_M062X}
% \end{align}

% \noindent where parameters $\alpha$ and $d_i$ ($i=0,\dots,5$) are empirically
% determined.

The exchange and correlation contributions to M06-2X are combined from VSXC and
M05 functional forms. The total exchange-correlation energy is expressed as:

\begin{align}
  E_{XC}^{\mathrm{M06-2X}} = \frac{54}{100}E_X^{\HF} + \left(1-\frac{54}{100}\right)E_X^{\mathrm{DFT}} + E_C^{\mathrm{DFT}},
\end{align}

\noindent with the correlation energy split into opposite-spin ($\alpha\beta$)
and parallel-spin ($\sigma\sigma$) components:

\begin{align}
  E_C^{\alpha\beta} &= \int \varepsilon^{\mathrm{GEH}}_{\alpha\beta}
    \left[g_{\alpha\beta}(x_\alpha,x_\beta) + h(x_{\alpha\beta},z_{\alpha\beta})\right]\dd\tau, \\
  E_C^{\sigma\sigma} &= \int \varepsilon^{\mathrm{GEH}}_{\sigma\sigma}
    \left[g_{\sigma\sigma}(x_\sigma) + h(x_\sigma,z_\sigma)\right]\dd\tau.
\end{align}

This detailed construction allows M06-2X to reliably model chemically complex
systems where accurate treatment of electron correlation is essential.






