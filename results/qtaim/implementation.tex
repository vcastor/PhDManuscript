% -*- coding: utf-8 -*-
\newpage
\section{Original Implementation}

\vspace{0.5cm}%
\begin{wrapfigure}[19]{r}{0.6\textwidth}
  \centering
  \resizebox{\linewidth}{!}{% -*- coding: utf-8 -*-
\begin{tikzpicture}[node distance=2cm, every node/.style={font=\scriptsize}]

  % Nodes
  \node (start)    [startstop, align=center]
    {The ATS for each\\ atom is generated};
  \node (pickgp)   [process, below of=start]
    {Pick up a new GP as initial point in the stepest ascent path};
  \node (satisfy?) [decision, below of=pickgp, yshift=-1.0cm, align=center]
    {Does the GP satisfy\\ screening conditions?};
  \node (setapp)   [process, below of=satisfy?, yshift=-0.7cm]
    {APP = GP};
  \node (inATS?)   [decision, below of=setapp]
    {Is the APP inside an ATS?};
  \node (addABP)   [process, right of=inATS?, xshift=3.8cm]
    {Add GP to ABP};
  \node (newNNA?)  [decision, below of=inATS?, yshift=-0.7cm]
    {Is the APP a new NNA?};
  \node (addNNA)   [process, right of=newNNA?, xshift=6.1cm, align=center]
    {Add NNA to the list of attractors with an ATS radius of 0.15};
  \node (forward)  [process, below of=newNNA?, yshift=-0.7cm]
    {Go a step forward in the stepest ascent path};

  % Arrows
  \draw [arrow] (start) -- (pickgp);
  \draw [arrow] (pickgp) -- (satisfy?);
  \draw [arrow] (satisfy?) -- node[anchor=south] {yes} ++(4.0,0) |- (pickgp.350);
  \draw [arrow] (satisfy?) -- node[anchor=east] {no} (setapp);
  \draw [arrow] (setapp) -- (inATS?);
  \draw [arrow] (inATS?) -- node[anchor=south] {yes} (addABP);
  \draw [arrow] (addABP) |- (pickgp);
  \draw [arrow] (inATS?) -- node[anchor=east] {no} (newNNA?);
  \draw [arrow] (newNNA?) -- node[anchor=south] {yes} (addNNA);
  \draw [arrow] (addNNA) |- (pickgp.10);
  \draw [arrow] (newNNA?) -- node[anchor=east] {no} (forward);
  \draw [arrow] (forward) -| ++(-3.5,0) |- (inATS?);

\end{tikzpicture}

}
  \caption{Flowchart of Rodríguez' original \gls{QTAIM} partition
    algorithm in \ams. The next acronyms are used in the flowchart:
    ATS, atomic trust sphere; GP, grid point;
    APP, actual point in the path; NNA, non-nuclear attractor;
    and ABP, atomic basin property.}
  \label{rodrigez_flowchart}
\end{wrapfigure}

The \gls{QTAIM} partition was originally implemented in \ams by
Rodríguez~\cite{Rodrguez2008}, enabling the computation of atomic properties
without explicitly solving the zero-flux condition
(Figure~\ref{rodrigez_flowchart}). This approach is a key reason why the
\gls{QTAIM} partition in \ams is remarkably fast.

% \vspace{1cm}%

The analysis is performed by partitioning real space into a grid, either
user-defined or automatically set by \ams with a default spacing of 0.5~bohr. A
Newton-Raphson method is applied to locate the \glspl{CP} of
$\rho(\mathbf{r})$. Beyond this, the grid serves also to compute the atomic
properties, with each point contributing to the numerical integration within
its assigned atomic basin. Points close to the nuclei are directly assigned to
the corresponding basin through the atomic trust sphere (ATS) with a radius of
0.23~bohr~\cite{Rodrguez2008}, accelerating the process.

Conversely, points far from the molecule typically have electron densities too
low to be clearly assigned to any atom. Since their contribution to properties
is negligible, they are excluded from the analysis. This screening condition is
applied when the density and its gradient satisfy the conditions: $i$)
$w_i\rho(\mathbf{r}_i) < 10^{-8}$ and $ii$)
$|\boldsymbol{\nabla}\rho(\mathbf{r}_i)|^2 < 10^{-8}$, where $w_i$ is the
integration weight associated with grid point $\mathbf{r}_i$.

\newpage

\vspace{0.6cm}%
Although the implementation successfully identified \glspl{CP} and
atomic basins, it did not provide a complete topological analysis. The
limitations of the original code are discussed in the following
subsections, alongside the description of the new implementation. A
summary of the updates introduced in the current version of the \ams
code is provided in the next Table.

\vspace{0.3cm}%
\begin{table}[h]
  \caption{Summary of updates introduced in the current version of the \ams code.}
  % -*- coding: utf-8 -*-
%makecell

\begingroup
\setstretch{0.95}
\begin{tcolorbox}[tab2,
  tabularx={>{\arraybackslash}m{2.5cm}|>{\arraybackslash}X|
            >{\arraybackslash}m{4.5cm}|>{\arraybackslash}m{4cm}},
  title=Comparison of Features in AMS,
  fontupper=\tiny,
  fonttitle=\bfseries,
  boxrule=0.5pt,
  ]

  & & & \\
  \textbf{Feature} & \textbf{First Implementation} &
  \textbf{Current Implementation} & \textbf{Comments} \\ \hline\hline

  & & & \\ [-2mm]
  \glspl{CP} & All detected &
  Code refactored and memory optimisation & Same results \\
  & & & \\ [-2mm] \hline

  & & & \\ [-2mm]
  \glspl{NNA} & Detected & Detected and analysed &
  Bonds to NNA allowed; atomic properties are computed \\
  & & & \\ [-2mm] \hline

  & & & \\ [-2mm]
  Gradient paths &
  \makecell[l]{Implemented by\\ Runge-Kutta 2$^\text{nd}$ Order} &
  \makecell[l]{Adaptive use of Runge-Kutta 2$^\text{nd}$\\
  or 4$^\text{th}$ Order with dynamic step \\ size
  }&
  Improves numerical stability and precision \\
  & & & \\ [-2mm] \hline

  % Gradient paths &
  % Implemented by Runge-Kutta 2$^\text{nd}$ Order &
  % Adaptive use of Runge-Kutta 2$^\text{nd}$
  % or 4$^\text{th}$ Order with dynamic step size &
  % Improves numerical stability and precision \\ \hline

  & & & \\ [-2mm]
  \glspl{RCP} \& \glspl{CCP} & \glspl{CP} detected &
  Atoms involved in rings/cages identified &
  Implemented gradient paths and graph theory \\
  & & & \\ [-2mm] \hline

  & & & \\ [-2mm]
  \makecell[l]{
  Atomic properties\\ by the form\\ 
  $\int_{\Omega}\rho_{A}(\mathbf{r})\dd\tau$
  } &
  Implemented &
  Refactored & Cleaner code structure; easier to read and maintain \\
  & & & \\ [-2mm] \hline

  % Atomic properties by the form
  % $\int_{\Omega}\rho_{A}(\mathbf{r})\dd\tau$ &
  % Implemented &
  % Refactored & Cleaner code structure; easier to read and maintain \\ \hline

  & & & \\ [-2mm]
  Dipole moment & Electronic contribution only &
  Nuclear contribution added; total dipole compared with molecular value &
  Full dipole moment computed \\
  & & & \\ [-2mm] \hline

  & & & \\ [-2mm]
  Polarisability & Not implemented &
  Available via \plams; new \ams driver task planned & \ams integration ongoing \\
  & & & \\ [-2mm] \hline

  % \rule{0pt}{6ex} 
  & & & \\ [-2mm]
  Excited states & Not implemented &
  Implemented & Pending final merge to trunk \\ %\hline
  & & & \\ [-2mm] \hline

  & & & \\ [-2mm]
  Atomic basins in GUI & Implemented &
  New GUI implementation & Improved visualisation \\
  & & & \\

\end{tcolorbox}
\endgroup


  \label{newVersion}
\end{table}

