% -*- coding: utf-8 -*-
\subsubsection{Rings as cycles and cages as convex polyhedra}\label{graph_theory}

For rings, the analysis determines whether the set of bonds forms a closed
cycle. This is assessed by examining the correspondence between vertices
(atoms) and edges (bonds): in a valid cycle, the number of vertices must equal
the number of edges. This condition provides a reliable and computationally
efficient means of detecting topological inconsistencies, and is therefore used
as a trigger for recomputing the topology when necessary. The procedure for
this verification is summarised in Algorithm~\ref{ring_algo}, which also
illustrates how the algorithm detects the presence of extra atoms bonded to the
ring, as exemplified in Figure~\ref{extra_atom}.

Cages present a greater challenge, as their topology is inherently more complex
and cannot be validated by a single criterion. As discussed previously, several
warning checks have been implemented, one of which adopts a purely geometrical
perspective: Euler's polyhedron formula. This relation links the number of
faces ($F$), edges ($E$), and vertices ($V$) of a polyhedron through the
expression $F - E + V = 2$. Compliance with this formula confirms that the cage
exhibits at least a topologically consistent convex polyhedral structure.
However, this condition alone does not guarantee physicochemical validity, and
additional topological and geometrical criteria may be necessary to achieve a
complete verification.

\begin{algorithm}
  \SetNoFillComment

  \tcc{Number of bonds between the atoms}
  nbonds = 0; found = false\;
  \For{i $\gets 1$ \KwTo ring size}{
    atomA $\leftarrow$ atomRing(i)\;
    \For{j $\gets i+1$ \KwTo ring size}{
      atomB $\leftarrow$ atomRing(j)\;
      \uIf{atomA and atomB are bonded}{
        nbond + 1\;
      }
    }
  }

  \uIf{nbonds = ring size}{
    info = true\;
  }
  \tcc{Any extra atom bonded to the ring?}
  \uIf{info $\mathbf{and}$ first time called}{
    delatom(:) $\leftarrow$ false \tcp*{Boolean array}
    \For{i $\gets 1$ \KwTo ring size}{
      atomA $\leftarrow$ atomRing(i)\;
      nbond = 0\;
      \tcc{All against all}
      \For{j $\gets 1$ \KwTo ring size}{
        atomB $\leftarrow$ atomRing(j)\;
        \If{atomA and atomB are bonded}{
          nbond + 1\;
        }
      }
      \tcc{Just one bond means the atom is not in the cycle/ring}
      \If{nbond < 2}{
        delatom(i) = true\;
      }
    }

    \tcc{Delate the extra atom}
    \uIf{any(delatom)}{
      j = 0\;
      \For{i $\gets 1$ \KwTo ring size}{
        \uIf{not delatom(i)}{
          j = j + 1\;
          tmp(j) $\leftarrow$ atomRing(i)\;
        }
      }
      ring size $\leftarrow$ j\;
      atomRing(:) $\leftarrow$ tmp(:)\;
    }
    \textbf{call me recursively}\;
  }

\caption{Algorithm to check the rings.}
\label{ring_algo}
\end{algorithm}

