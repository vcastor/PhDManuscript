\newpage
\thispagestyle{empty}
\newpage
\ % Emtpy page
\newpage
\thispagestyle{empty}

\vspace*{0.2cm}%

\noindent \textbf{Résumé :} Ce travail vise à approfondir la compréhension de la
réactivité chimique en solution à travers l'intersection de la chimie théorique
et computationnelle.  Il présente une approche hybride de la réactivité
chimique en solution fondée sur la DFT conceptuelle et QTAIM. L'objectif
central est de développer des outils permettant une meilleure compréhension des
facteurs électroniques et thermodynamiques qui gouvernent la réactivité, tout
en tenant compte des effets de solvatation.

Trois chapitres principaux ont été développés pour atteindre cet objectif.  Le  
premier est consacré à l'amélioration et à l'extension de la partition QTAIM
dans la suite de programmes Amsterdam Modeling Suite. Le deuxième chapitre
porte sur le développement d'un modèle prédisant la décroissance de l'entropie
en solution par rapport au gaz parfait. Le troisième chapitre est consacré à
la prédiction des aspects thermodynamiques et cinétiques de la nucléophilicité
en solution à l'aide des descripteurs de DFT conceptuelle et de QTAIM.

\noindent \textbf{Mots-clés :} DFT conceptuelle, QTAIM, Entropie, Nucléophilicité

\vspace{1.2cm}%

\noindent \textbf{Abstract:} This work aims to advance the understanding of chemical
reactivity in solution through the intersection of theoretical and
computational chemistry. It presents a hybrid approach to chemical reactivity
in solution based on Conceptual DFT and QTAIM. The central goal is to develop
tools that provide a deeper understanding of the electronic and thermodynamic
factors governing reactivity, while accounting for solvation effects.

Three main chapters were developed to achieve this goal. The first
one is dedicated to the improvement and extension of the QTAIM partition in the
Amsterdam Modeling Suite (AMS). The second chapter focuses on the development
of a model that predicts the decay of entropy in solution with respect to the
ideal gas. The third chapter is dedicated to the prediction of thermodynamic
and kinetic aspects of nucleophilicity in solution using Conceptual DFT and
QTAIM descriptors.

\noindent \textbf{Keywords:} Conceptual DFT, QTAIM, Entropy, Nucleophilicity
