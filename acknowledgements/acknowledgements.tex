%
% Acknowledgements,
%   la parte más linda de la tesis,
%     感谢
%

%
% i'm an arrogant bastard,
%   i know it,
%     and i don't care.
%
\newpage
\ % The empty page
\begin{flushright}
  \glqq New thesis,\phantom{\grqq.}\\
  Same old mistakes\grqq.
\end{flushright}
\vspace*{7cm}
``Dear Sir or Madam,\\ [-0.5em]
\hspace*{0.4\textwidth}\makebox[0pt][r]{will you read my thesis?}\\ [-0.5em]
\phantom{``}it took me years to write, \\ [-0.5em]
\hspace*{0.4\textwidth}\makebox[0pt][r]{will you take a look?}''\\ [-0.5em]
\vspace*{7cm}
\begin{flushright}
  <<I've got nothing to say,\phantom{.''}\\
  but it's okay.>>
\end{flushright}

\newpage



\vspace*{1.0cm}
{\calligra\textbf{\Huge{Acknowledgements}}}
\vspace*{1.8cm}

{\setmainfont{QTChanceryType}\itshape{

  I'd like to express all my gratitude to all those people who have helped me in
  one way or another to reach the point where i am today. They all ---with or
  without the best intentions--- have contributed to creating the person i am,
  and therefore, to the work i present today.

  \vspace{5mm}%
  Je veux remercier l'Université de Rouen de m'avoir offert l'opportunité de
  réaliser mon doctorat, ainsi que le centre CRIANN pour le temps de calcul et
  l'infrastructure mise à disposition. J'exprime aussi ma gratitude à tous les
  membres du jury ---rapporteurs, examinateurs et directeurs---, non seulement
  pour le temps consacré à évaluer ce travail, mais aussi pour l'intérêt et la
  confiance qu'ils ont portés à ma recherche.

  \vspace{5mm}%
  Merci à Laurent Joubert et Vincent Tognetti pour leur encadrement et leur
  patience tout au long de cette thèse. À Julien Legros et Philippe Jubault,
  directeur et directeur-adjoint de l'Institut CARMeN, 
  Frédéric Guégan, Julien Pilmé, Isabelle Chataigner et Aurélien Moncomble, merci
  pour le temps consacré à la relecture de ce manuscrit,

  \noindent\hspace*{3.5\parindent} ... y particularmente gracias a Toche (José Manuel
  % \hfill... y particularmente gracias a Toche (José Manuel
  Guevara Vela).

  \vspace{5mm}%
  Software for Chemistry and Materials, bedankt! (et takk), it was a real
  pleasure to work with you.  Being there during the long-awaited office move
  from the W\&N to the OZW building, after years of hearing ``we're moving in a
  few weeks''.
  A bit of organised chaos
  gave rhythm to the days, het was gezellig.

  \vspace{5mm}%
  \noindent\hspace*{6\parindent}À mes collègues de recherche d'hier et d'aujourd'hui,\\
  \noindent\hfill never forgetting the one who is not physically with us,\\
  \noindent\hspace*{6\parindent}but always in all research groups,\\
  \noindent\hfill{\crylit{Aleksándra \'{Z}lbak\'{r}n}} (\textit{Alexandra Elbakyan}).

  \newpage
  \vspace*{4.8cm}

  \vspace{5mm}% La familia en español
  Cinco años viviendo en otro continente, catorce (o más) horas de viaje de por
  medio u ocho horas de diferencia horaria no me han separado de mi núcleo
  familiar. Libia, Victor y Ana habéis sido un pilar en mi vida, sin ustedes no
  hubiera sido posible llegar hasta aquí...

  \noindent\hspace*{2\parindent}
    it's been a long ride since my bachelor's days, isn't it?\\ [-2pt]
  \hspace*{0.74\textwidth}\makebox[0pt][r]{
    Pau \emoji{sauropod} LC and Andrea \emoji{carrot} FL,}\\ [-5pt]
  \noindent\hspace*{4\parindent}
    idk how but you are not tired of me (yet).\\ [-5pt]
  % \hspace*{0.74\textwidth}\makebox[0pt][r]{
  %   aunque tampoc sé como Felipe me soporta.}\\ [-18pt]

  \vspace{11mm}%
  And yeah... how could i forget the adventures that coloured this PhD? Skiing in
  Norway, forgetting my passport going to Barcelona, getting lost in Paris
  a thousand times with Trinidad, getting sick in Köln with Pau...  i've lost
  count of my trips to Madrid, and of all the huge tiny things that made this
  time unforgettable:
  Amsterdam, Wien, Mexico...\\ [-0.2em]
  \vspace{-6pt}%
  \hspace*{0.88\textwidth}\makebox[0pt][r]{
    et bien sûr, Rouen ---tu...
    ma petite ville grise et pluvieuse,}\\
  \vspace{-7pt}%
  \hspace*{0.88\textwidth}\makebox[0pt][r]{
    tu es aussi réconfortante}---\\
  \vspace{-12pt}%
  \hspace*{0.88\textwidth}\makebox[0pt][r]{
    merci d'avoir été
    un chez moi quand j'en avais besoin}.

}}

\hspace*{0.9\textwidth}\makebox[0pt][r]{
\begin{CJK*}{UTF8}{gkai}
  \noindent\hspace*{2.5\parindent}\vvv{非常感谢}
\end{CJK*}
}

\vspace{-15pt}%
\noindent\hspace*{2\parindent}{\HUGE \emoji{sparkles} \textsubscript{\emoji{sparkles}}}\\

\vspace{-1.1pt}%
\hspace*{0.8\textwidth}\makebox[0pt][r]{
  \vcastor[\number\year]
}

\newpage

