% Notation

\newglossaryentry{iunit}{
  type=notation,
  name={\ensuremath{i}},
  description={Imaginary unit, square root of -1, \ensuremath{\sqrt{-1}}}
}

\newglossaryentry{hbar}{
  type=notation,
  name={\ensuremath{\hbar}},
  description={Reduced Plank's constant, \ensuremath{\nicefrac{h}{2\pi}}}
}

\newglossaryentry{hamiltoniano}{
  type=notation,
  % name={\Ha},
  name={\ensuremath{\widehat{H}}},
  description={Hamiltonian operator}
}

\newglossaryentry{nablacuadrada}{
  type=notation,
  name={\ensuremath{\boldsymbol{\nabla}^2}},
  description={Laplace operator, \ensuremath{\nabla\cdot\nabla}}
}

\newglossaryentry{cspeed}{
  type=notation,
  name={\ensuremath{c}},
  description={Speed of light in the vacuum}
}

\newglossaryentry{gconstant}{
  type=notation,
  name={\ensuremath{G}},
  description={Gravitational constant}
}

\newglossaryentry{kb}{
  type=notation,
  name={\ensuremath{k_B}},
  description={Boltzmann constant}
}

\newglossaryentry{Psi}{
  type=notation,
  name={\ensuremath{\Psi}},
  description={Wavefunction}
}

\newglossaryentry{Psiket}{
  type=notation,
  name={\ensuremath{\ket{\Psi}}},
  description={Ket of wavefunction, braket notation also called Dirac notation}
}

\newglossaryentry{pi}{
  type=notation,
  name={\ensuremath{\pi}},
  description={number \ensuremath{\pi}}
}

% Joke [Spanish notation for combinatoria] ¡n!
\newglossaryentry{nchoosm}{
  type=notation,
  name={\ensuremath{\binom{n}{m}}},
  description={Binomial coefficient, ``n choose m'', 
  \ensuremath{\frac{n!}{k!(n-k)!}}}
  % or in Spanish: \text{¡}n!(\text{¡}m!\text{¡}n-m!)$^{-1}$}
    % or in Spanish: \ensuremath{\frac{\text{¡}n!}{\textexclamdown k!\text{¡}n-k!}}}
    % or in Spanish: $\frac{¡n!}{¡m!¡n-m!}$}
    % description={Binomial coefficient, "n choose k", combinatorial notation,
    % \ensuremath{\frac{n!}{k!(n-k)!}}\\
    % or in Spanish: \ensuremath{\frac{¡n!}{¡k!¡n-k!}}}
}

% \newglossaryentry{vecnote}{
%   type=notation,
%   name={\ensuremath{\mathbf{v}}},
%   description={In this manuscript the vectors are denoted mainly by boldface letters, \\
%                an arrow on top of the letter in some cases, \ensuremath{\mathbf{v} = \vec{v}}, \\ they
%                two refer to the elements from an Abelian Group in a Vector Space.}
% }

% \glsadd{vecnote}
\glsadd{nchoosm}

% \renewcommand*{\glossarypreamble}{Throughout this thesis, all integrals are
% understood as Riemann sums. For clarity and simplicity, differentials have been
% omitted in some cases where their inclusion would be redundant or implicitly
% understood.}


