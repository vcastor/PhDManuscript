% -*- coding: utf-8 -*-
\newpage
\section{Variational Method}\label{variational_method}

It is essential to highlight the importance of the
Variational Method, particularly in the context of the Schrödinger Equation,
which is fundamentally an eigenvalue problem. Given this, it is possible to
apply techniques developed for eigenvalue problems, such as the variational
method, to obtain approximate solutions. Specifically, this method can be used
to approximate eigenvalues in the general form: $\widehat{O}\varphi =
\omega\varphi$. This approach is commonly employed to compute the energy of
atomic and molecular systems with $N$ electrons, providing a practical way to
address complex quantum mechanical problems.

The minimization of the electronic energy from a wavefunction serves as the
foundation of the variational method. By optimizing the coefficients of the
expansion, we can obtain the ground state wavefunction.

\newpage

\follow{Variational Method}{

  Let $\widehat{A}$ be a Hermitian operator, and suppose there exists a finite
  set of exact solutions to the eigenvalue equation such that:

  \begin{align}
    \widehat{A}|\phi_{\alpha} \rangle =
    \epsilon_{\alpha}|\phi_{\alpha} \rangle \qquad with:\ \epsilon_{0} \le
    \epsilon_{1} \le \ldots \le \epsilon_{\alpha} \le \ldots
    \label{val_schr}
  \end{align}

  If we assume that the set $\epsilon_{\alpha}$ consists of discrete values,
  and that the eigenfunctions are orthonormal, then by applying
  $\langle\phi_{\beta}|$ on the left-hand side of Equation \ref{val_schr}, we
  obtain:

  \begin{align}
    \langle\phi_{\beta} |\widehat{A} | \phi_{\alpha}\rangle =
    \epsilon_{\alpha}\delta_{\alpha\beta}
  \end{align}

  Since the eigenfunctions of $\widehat{A}$ form a complete set, any function
  $\varphi$ can be expressed as a linear combination of $\phi_{\alpha}$,
  provided that $\varphi$ satisfies the boundary conditions of the
  system~\cite{szabo}.

  \begin{align}
    |\varphi\rangle = \sum_{\alpha}|\phi_{\alpha}\rangle c_{\alpha} = 
    \sum_{\alpha}|\phi_{\alpha}\rangle\langle\phi_{\alpha} |\varphi\rangle,
  \end{align}
  and
  \begin{align}
    \langle\varphi| = \sum_{\alpha}c^{\star}_{\alpha} \langle\phi_{\alpha}|= 
    \sum_{\alpha} \langle\varphi | \phi_{\alpha} \rangle \langle\phi_{\alpha}
  \end{align}

  The variational theorem states that for a system where $i$) the Hamiltonian is
  time-independent, $ii$) the ground state eigenvalue is $\varepsilon_0$ and
  $iii$) the normalized function $\varphi$ satisfies the boundary conditions,
  then:

  \begin{align}
    \langle \varphi |\Ha | \varphi \rangle \ge \varepsilon_{0}
  \end{align}

}

\newpage

