% -*- coding: utf-8 -*-
\newpage
\section{How this thesis was written}\label{how_written}

This thesis was written using \LaTeX, bibliography management handled by
\textsc{Bib}\TeX{}. The compilation was done via Lua\LaTeX, using a
\texttt{Makefile} that automates the compilation process as well
as some of the imagen generation.

Figures were generated using \python (Matplotlib and Seaborn), and Ti$k$Z. Image
editing was carried out using GIMP and Inkscape. Various diagrams were created
with Graphviz and Mermaid, this last one for the README file. For the vector
graphics, the \href{https://www.svgrepo.com}{svgrepo} was used to obtain some
starting points, and then modified with Inkscape.

For the Figure~\ref{awaflux}, a modified version of \textsc{qtaim.wl} was used,
which is a \textsc{Wolfram language} script, to analyse the QTAIM partition
from \texttt{wfn} files. The original code is available at
\href{http://www.github.com/ecbrown/QTAIM.wl}{\faGithub/ecbrown/QTAIM.wl}. The
original code is quite experimantal, and my modificated version is even more
unestable, no guarantees are made about the generalised version of the code.

Any issue or \bug will be tracked in the repository. However,
since the code functions as intended for this thesis, no further development is
currently planned. All warnings i have during \emph{local compilation} were
analysed and just ignored, as they do not affect the final result.

The project is intended to be used from the command line. Nonetheless, due to
the widespread use of Overleaf, a dedicated strategy is provided to make it
compatible. Note that Overleaf does not support all features of the original
setup, such as \texttt{Makefile} options (compaling just one Chapter or don't compilation for
bibliography) or multi-branch workflows, typographies
are limited and the compilation time would be longer since many images/plots
are complex TikZ figures.

The Overleaf branch should be used mainly for minor text edits. It is
recommended to treat it as a mirror of the main \texttt{writing}
branch. To synchronize changes, we suggest using selective cherry-picks of
\texttt{*.tex} files to avoid merge conflicts.

\noindent\textit{Important:} Overleaf does not allow privileged \git operations
such as \texttt{git~push~--force}, which prevents proper handling of merges
with unrelated histories. To initialise the Overleaf branch, i recommend,
the following steps:

\coffeestainA{0.2}{0.65}{90}{-7cm}{11cm}
\newpage

\lstset{style=terminal, numbers=none,escapeinside={(*@}{@*)}}
\begin{macterminal}[vcastor@ada - bash]
$ # From Overleaf track everything
$ git pull
$ # Delete all Overleaf files via Overleaf UI or locally
$ # Then merge with unrelated histories allowed:
$ git merge writing –allow-unrelated-histories
$ git commit -am "My first commit on Overleaf"
$ git push
\end{macterminal}

The full source code of this manuscript is available in a public GitHub
repository:
\href{http://www.github.com/vcastor/PhDManuscript}{\faGithub/vcastor/PhDManuscript}.

My recommendation is to use the \terminal to compile the manuscript, as this
project was designed, to compile in local you can follow the next steps:

\lstset{style=terminal, numbers=none,escapeinside={(*@}{@*)}, moredelim=[is][\color{white}]{§}{§}}
\begin{macterminal}[vcastor@ada - bash]
$ # Download the repository
$ git clone https://github.com/vcastor/PhDManuscript.§git§
$ cd PhDManuscript
$ # Compile the entire manuscript
$ make all  # imagenes, plots, bibliography and text
$ # Or just what you need
$ make tex  # no imagenes or plots but bibliography
$ make fast # no imagenes or plots neither bibliography
$ make chapter # just the chapter [READ THE README.md for that]
$ make style   # style template
\end{macterminal}
\lstset{style=mystyle}

\coffeestainC{0.3}{0.65}{-90}{6.5cm}{-6.5cm}

