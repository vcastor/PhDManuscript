% -*- coding: utf-8 -*-
\newpage
\section{System's Coordinates}\label{xyzfiles}

Throughout this thesis, various molecular systems were computed both to
analyse their intrinsic properties and to validate our implementation.
For brevity, the Cartesian coordinates of these systems have been omitted
from the main chapters. They are instead collected here in \texttt{xyz}
format for reference and reproducibility. All coordinates are reported in
\SI{}{\angstrom}.

For the case of the Be$_3^{-2}$ system, the geometries can be generated
using the code example provided in Section~\ref{ams_section}. In that
example, a series of equilateral triangles is constructed with side lengths
of 2.700, 2.200, 2.150, 2.120, 2.090, 2.070, 1.960, and 1.800~\AA.

For the set of $\sim$45k systems used to compare the dipole moment
implementation, the full dataset ($\sim$180~MB) is available upon request from
the author.

\vspace*{1cm}%
{\renewcommand{\baselinestretch}{.5}
\scriptsize{

\noindent \ce{($\eta^5$-C$_5$H$_5$)$_2$Fe} Ferrocene
\VerbatimInput{./appendix/xyz/ferrocene.xyz}

\noindent \ce{C_10H_4} Deformated C-ring
\VerbatimInput{./appendix/xyz/c10.xyz}

\newpage
\noindent \ce{gdb\_str\_}10035 Extra atom in a ring
\VerbatimInput{./appendix/xyz/gdb_str_10035.xyz}

\noindent \ce{gdb\_str\_}10410 A CCP between two rings
\VerbatimInput{./appendix/xyz/gdb_str_10410.xyz}

\noindent \ce{Dicyclopentadiene} Transition state
\VerbatimInput{./appendix/xyz/dats.xyz}

}}
    
      
