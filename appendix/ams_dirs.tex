% -*- coding: utf-8 -*-
\newpage
\section{AMS Directory Structure}

The \ams driver is a large project composed of many interdependent
components, and its directory structure reflects this complexity. For the
benefit of any future developer reading this document, this section provides a
concise overview of the organisation of the \ams directory tree, outlining
the purpose of its main folders.

\begin{figure}[h]
\centering
  \begin{tikzpicture}[
    dirtree,
    % Even less indentation
    level 1/.style={sibling distance=0.7cm, level distance=1.5cm},
    level 2/.style={sibling distance=0.7cm, level distance=1.3cm},
    level 3/.style={sibling distance=0.7cm, level distance=1.1cm},
  ]
    % Define a vertical line for comment alignment (optional visual guide)
    % \draw[gray!20, dashed] (7,-1) -- (7,-20);
    
    \node[directory] (root) {\$AMSHOME/}
      child { node[directory] (atomicdata) {atomicdata/} }
      child { node[directory] (data) {data/} }
      child { node[directory] (examples) {examples/} }
      child { node[directory] (ext) {ext/}
        child { node[directory] (ftl) {ftl/} }
      }
      child[missing] {}
      child { node[directory] (gui) {gui/} }
      child { node[directory] (install) {Install/} }
      child { node[directory] (scripting) {scripting/}
        child { node[directory] (scm) {scm/}
          child { node[directory] (plams) {plams/} }
          child { node[directory] (dots1) {...} }
        }
      }
      child[missing] {}
      child[missing] {}
      child[missing] {}
      child { node[directory] (src) {src/}
        child { node[directory] (ams) {ams/}
          child { node[directory] (applications) {applications/} }
          child { node[directory] (engines) {engines/} }
        }
        child[missing] {}
        child[missing] {}
        child { node[directory] (adf) {adf/} }
        child { node[directory] (band) {band/} }
        child { node[directory] (lib) {lib/}
          child { node[directory] (base) {base/} }
          child { node[directory] (core) {core/} }
          child { node[directory] (dft) {dft/} }
        }
        child[missing] {}
        child[missing] {}
        child[missing] {}
      }
      child[missing] {}
      child[missing] {}
      child[missing] {}
      child[missing] {}
      child[missing] {}
      child[missing] {}
      child[missing] {}
      child[missing] {}
      child[missing] {}
      child { node[directory] (userdoc) {userdoc/} };
      
    % Add all comments aligned at x=7
    \node[comment] at (7,0 |- atomicdata) {things like basis sets};
    \node[comment] at (7,0 |- data) {JSON files for input definition};
    \node[comment] at (7,0 |- examples) {test};
    \node[comment] at (7,0 |- ext) {external dependencies for AMS};
    \node[comment] at (7,0 |- gui) {All the GUI code};
    \node[comment] at (7,0 |- install) {build system and precompiled things};
    \node[comment] at (7,0 |- scripting) {all Python code};
    \node[comment] at (7,0 |- plams) {our main Python package};
    \node[comment] at (7,0 |- src) {Fortran and C++ code};
    \node[comment] at (7,0 |- ams) {the AMS driver};
    \node[comment] at (7,0 |- applications) {applications for the AMS driver};
    \node[comment] at (7,0 |- engines) {small engines};
    \node[comment] at (7,0 |- adf) {big engines like ADF ...};
    \node[comment] at (7,0 |- band) {... and BAND get their own directory};
    \node[comment] at (7,0 |- lib) {library code};
    \node[comment] at (7,0 |- base) {basic things in shared lib};
    \node[comment] at (7,0 |- core) {more advanced things in static lib};
    \node[comment] at (7,0 |- dft) {shared code for ADF and BAND};
    \node[comment] at (7,0 |- userdoc) {documentation for end-users};
  \end{tikzpicture}
  \caption{AMS directory structure with perfectly aligned comments.}
\end{figure}

