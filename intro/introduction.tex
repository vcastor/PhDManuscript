% -*- coding: utf-8 -*-
\chapter{Introduction}

Comprendre et prédire la réactivité chimique en solution constitue un défi
majeur en chimie théorique. Cette réactivité dépend fondamentalement de deux
composantes : la contribution électronique, gouvernée par la structure
électronique des molécules, et la faisabilité thermodynamique, déterminée par
l'énergie de Gibbs $G = E_{\mathrm{electronic}} + \Delta H - TS$, qui combine
l'énergie électronique, les corrections thermiques et
l'entropie~\cite{scmGibbs2025}.

Dans ce contexte, l'analyse et l'interprétation des descripteurs quantiques
s'avèrent indispensables pour la compréhension des systèmes chimiques.  La
densité électronique, les orbitales moléculaires, les descripteurs issus de la
DFT conceptuelle (\gls{CDFT})~\cite{Parr1995}, ainsi que les topologies
dérivées de la théorie quantique des atomes dans les molécules
(\gls{QTAIM})~\cite{bader}, constituent autant d'outils complémentaires
permettant de décrypter la réactivité chimique.  Toutefois, la synergie entre
ces approches, et en particulier l'intégration des informations topologiques et
conceptuelles, reste encore peu explorée à ce jour.

Ce travail propose une approche intégrée, alliant QTAIM et CDFT, afin d'accéder
à une représentation multiscalaire de la réactivité. La QTAIM permet une
partition rigoureuse de l'espace moléculaire en régions atomiques, à partir
desquelles peuvent être définis des descripteurs localisés ou condensés tels
que les charges, les moments dipolaires ou les polarisabilités. Ces grandeurs
sont directement liées à la topologie de la densité électronique, et offrent
une lecture spatialisée du comportement chimique.

\newpage
La DFT conceptuelle, de son côté, propose un cadre théorique puissant pour
l'analyse de la réactivité chimique à travers des descripteurs variés. Elle
permet de définir, au moyen de la densité électronique, des indices globaux,
locaux et non-locaux de réactivité, tels que le potentiel chimique electronique,
entre autres. L'intégration de ces descripteurs au sein d'un cadre
\gls{QTAIM} offre la possibilité de concilier une analyse énergétique et structurelle,
tout en conservant une interprétabilité chimique forte. Cette approche hybride
a été implémentée dans le software Amsterdam Modeling Suite
(\ams)~\cite{ADF2001}, avec de nouveaux modules développés au cours de cette
thèse.

La dimension thermodynamique est également prise en compte via une modélisation
conformationnelle explicite. En effet, les propriétés mesurées
expérimentalement correspondent souvent à une moyenne sur plusieurs
conformères, particulièrement en phase condensée. Pour en rendre compte, notre
méthodologie combine un échantillonnage rapide basé sur le
DFTB~\cite{ElstnerSeifert2014} à une réoptimisation DFT~\cite{Hohenberg1964} de
haute précision, suivie d'une moyenne pondérée par la distribution de
Boltzmann~\cite{Rowlinson2005}. Ce protocole permet d'obtenir des descripteurs
réalistes, intégrant les effets entropiques et conformationnels.

Dans certains cas, notamment pour l'étude de systèmes biologiques ou de
réactions en milieu hétérogène, il est nécessaire d'avoir recours à des
approches multi-échelles. Le formalisme mécanique quantique/mécanique
moléculaire et en particulier le schéma de Polarisable
Embedding~\cite{Beerepoot2016}, permet de tenir compte de la polarisation
mutuelle entre une région traitée en mécanique quantique et un environnement
classique.  Ces méthodes nécessitent des descripteurs électrostatiques précis,
comme les charges et les polarisabilités atomiques, pour lesquels la QTAIM
constitue une base naturelle. Toutefois, les bases de données disponibles,
notamment via la bibliothèque PyFrame~\cite{zenodo}, sont encore limitées en
couverture chimique et en compatibilité avec les modèles avancés de
solvatation.

% \newpage

% Afin de surmonter ces limites, ce travail explore également l'utilisation de
% modèles d'apprentissage automatique. À partir de bases de données enrichies
% avec des agrégats de solvant, nous évaluons la capacité d'un modèle existant,
% NNAIMQ~\cite{Gallegos2022}, à prédire les charges QTAIM, et nous étendons cette
% approche à la prédiction de nouveaux descripteurs, comme les dipôles atomiques.
% Bien que le développement complet de modèles ML ne constitue pas l'objectif
% principal de ce travail, les outils mis en place jettent les bases pour une
% généralisation future, à la croisée de la chimie computationnelle et de la
% science des données.

En résumé, cette thèse s'inscrit dans une démarche de modélisation intégrée de
la réactivité chimique en solution, en développant des outils robustes pour
l'analyse électronique, la prise en compte des effets entropiques et
conformationnels, et l'exploitation des méthodes de solvatation. En conjuguant
la rigueur de \gls{QTAIM}, la pertinence des descripteurs \gls{CDFT}, et le potentiel de
l'apprentissage automatique, nous proposons une boîte à outils numérique
polyvalente, extensible, et physiquement interprétable, à même d'améliorer
l'analyse et la prédiction de la réactivité dans des milieux complexes.

% \newpage
%  -*- coding: utf-8 -*-
\section{Objectifs}

L'objectif principal de cette thèse est de développer une méthodologie
computationnelle intégrée pour l'analyse et la prédiction de descripteurs
chimiques en solution. Cette approche vise à concilier la rigueur des
fondements théoriques (\gls{QTAIM} et \gls{CDFT}) avec des techniques modernes
de modélisation moléculaire, dans un cadre compatible avec la chimie en phase
condensée.

Les objectifs spécifiques de ce travail sont les suivants :

\begin{itemize}
  \item Étendre l'analyse topologique à des éléments complexes de \gls{QTAIM},
        tels que les anneaux, cages et attracteurs
        non nucléaires, pour enrichir la description de la structure électronique.
  \item Développer et intégrer dans \ams des routines permettant
        le calcul de moments dipolaires et de polarisabilités atomiques dans le
        cadre de QTAIM à partir de la densité électronique issue de calculs DFT.
  \item Mettre en place un protocole combinant la vitesse \gls{DFTB} avec la
        précision \gls{DFT} pour l'échantillonnage conformationnel, et intégrer une
        moyenne thermodynamique via la distribution de Boltzmann afin de refléter les
        effets entropiques et statistiques.
  \item Mettre à profit des modèles d'apprentissage automatique, en utilisant
        comme variables d'entrée des descripteurs \gls{QTAIM} et \gls{CDFT}, pour
        prédire des propriétés chimiques telles que la nucléophilie.
  % \item Fournir des descripteurs localisés, compatibles avec les modèles de
  %       solvatation explicite et les approches multi-échelles comme le
  %       Polarisable Embedding, en vue d'une application à des systèmes
  %       biologiques ou complexes.
\end{itemize}

L'ensemble de ces objectifs vise à produire des outils numériques robustes,
interprétables, et transférables, capables de contribuer à la compréhension
fine de la réactivité chimique en solution et à son intégration dans des
modèles prédictifs à l'interface entre la théorie et l'expérience.



