%  -*- coding: utf-8 -*-
\section{Objectifs}

L'objectif principal de cette thèse est de développer une méthodologie
computationnelle intégrée pour l'analyse et la prédiction de descripteurs
chimiques en solution. Cette approche vise à concilier la rigueur des
fondements théoriques (\gls{QTAIM} et \gls{CDFT}) avec des techniques modernes
de modélisation moléculaire, dans un cadre compatible avec la chimie en phase
condensée.

Les objectifs spécifiques de ce travail sont les suivants :

\begin{itemize}
  \item Étendre l'analyse topologique à des éléments complexes de \gls{QTAIM},
        tels que les anneaux, cages et attracteurs
        non nucléaires, pour enrichir la description de la structure électronique.
  \item Développer et intégrer dans \ams des routines permettant
        le calcul de moments dipolaires et de polarisabilités atomiques dans le
        cadre de QTAIM à partir de la densité électronique issue de calculs DFT.
  \item Mettre en place un protocole combinant la vitesse \gls{DFTB} avec la
        précision \gls{DFT} pour l'échantillonnage conformationnel, et intégrer une
        moyenne thermodynamique via la distribution de Boltzmann afin de refléter les
        effets entropiques et statistiques.
  \item Mettre à profit des modèles d'apprentissage automatique, en utilisant
        comme variables d'entrée des descripteurs \gls{QTAIM} et \gls{CDFT}, pour
        prédire des propriétés chimiques telles que la nucléophilie.
  % \item Fournir des descripteurs localisés, compatibles avec les modèles de
  %       solvatation explicite et les approches multi-échelles comme le
  %       Polarisable Embedding, en vue d'une application à des systèmes
  %       biologiques ou complexes.
\end{itemize}

L'ensemble de ces objectifs vise à produire des outils numériques robustes,
interprétables, et transférables, capables de contribuer à la compréhension
fine de la réactivité chimique en solution et à son intégration dans des
modèles prédictifs à l'interface entre la théorie et l'expérience.

