% -*- coding: utf-8 -*-
\chapter{Conclusions}

This thesis has led to the development of a coherent and versatile
methodology for studying chemical reactivity in solution, grounded in the
tools of modern theoretical chemistry.

From a methodological standpoint, we have implemented new routines in the \ams
software to compute atomic dipole moments and polarisabilities, as well
as improved the calculation of local properties and extended their
implementation in QTAIM. These features enrich the set of available
descriptors, providing a local and physically interpretable perspective on
chemical reactivity.

We also proposed a protocol for modelling conformational effects
in solution. This protocol relies on conformer sampling followed
by statistical averaging based on the Boltzmann distribution. It has been
successfully applied to the study of thermodynamic properties such as entropy,
contributing to a more realistic and comprehensive description of reaction
mechanisms in the condensed phase.

In parallel, a structured database was constructed from results obtained via
\gls{QTAIM} and \gls{CDFT}, also including solvent aggregates. While the
development of machine learning models remains incomplete, the dataset
generated is already sufficient to support exploratory ML approaches and lays
the groundwork for future generalisation.

\newpage
Finally, a module dedicated to the study of excited states was developed
within the scope of this thesis, with a view towards future spectroscopic
applications. This development is functional but still requires finalisation
and validation.

In summary, this thesis lays the foundation for a complete and extensible
digital toolbox for studying chemical reactivity in solution. The avenues
opened ---especially those related to excited states and the integration of
machine learning techniques--- offer promising prospects for future work at the
interface between theory, simulation, and data modelling.

